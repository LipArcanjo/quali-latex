%%%%%%%%%%%%%%%%%%%%%%%%%%%%%%%%%%%%%%%%%%%%%%%%%%%%%%%%%%%%%%%%%%%%%%%%%%%%%%%%
%%%%%%%%%%%%%%%%%%%%%%% SUMÁRIO, CABEÇALHOS, SEÇÕES %%%%%%%%%%%%%%%%%%%%%%%%%%%%
%%%%%%%%%%%%%%%%%%%%%%%%%%%%%%%%%%%%%%%%%%%%%%%%%%%%%%%%%%%%%%%%%%%%%%%%%%%%%%%%

\makeatletter
% Formatação personalizada do sumário, lista de tabelas/figuras etc.
%\usepackage{titletoc}

% titlesec permite definir formatação personalizada de títulos, seções etc.
% Observe que titlesec é incompatível com os comandos refsection
% e refsegment do pacote biblatex!
% Vamos usar titlesec apenas
% para fazer títulos, seções etc. não serem justificados.
\usepackage[raggedright]{titlesec}

% Permite saber o número total de páginas; útil para colocar no
% rodapé algo como "página 3 de 10" com "\thepage\ de \zpageref{LastPage}"
%\usepackage{zref-lastpage,zref-user}

% Permite definir cabeçalhos e rodapés
\usepackage{fancyhdr}

% A formatação dos cabeçalhos/rodapés envolve duas coisas:
%
% 1. Definir qual texto deve ser impresso em cada página (nome
%    do capítulo ou seção, nome do autor, número da página etc.)
%
% 2. Imprimir esse texto no lugar correto (esquerda, direita ou
%    centro, cabeçalho ou rodapé)
%
% A definição do texto a imprimir é complexa, pois a preparação dos
% cabeçalhos/rodapés acontece de maneira independente da construção do
% restante da página. Textos fixos como, digamos, o nome do livro ou
% do seu autor, não trazem dificuldades. Já textos variáveis, como o
% nome do capítulo ou seção, o número da página etc. envolvem a
% passagem de informações sobre a página atual para o mecanismo de
% construção dos cabeçalhos.
%
% Isso é feito através de "variáveis" especiais chamadas "marks". O uso
% mais comum das marks é armazenar o nome do capítulo e da seção atuais:
% a cada novo capítulo ou seção, os comandos \chaptermark e \sectionmark
% são executados (isso é automático, você não precisa fazer nada); esses
% comandos definem uma "mark" com o nome do capítulo/seção a ser usado
% nos cabeçalhos/rodapés. Assim, modificando a definição dos comandos
% \chaptermark e \sectionmark, é possível modificar o texto a ser
% incluído nos cabeçalhos/rodapés.
%
% Definido o conteúdo das marks, o lugar em que cada texto é impresso
% é definido com os comandos \fancyhead e \fancyfoot. "RO" significa
% "Right side of Odd pages"; "LE" significa "Left side of Even pages" etc.
%
% LaTeX por padrão tem duas "marks": "\leftmark" e "\rightmark". Elas
% têm esses nomes porque originalmente eram usadas para armazenar o
% conteúdo a ser impresso nas páginas pares e ímpares. Isso, no entanto,
% não é obrigatório; a escolha desses nomes foi bastante infeliz. Para
% definir o conteúdo dessas marks, usam-se os comandos \markboth{A}{B}
% (define \leftmark como A e \rightmark como B) e \markright{C} (define
% \rightmark como C). Não há o comando \markleft.
%
% O procedimento de definição dos cabeçalhos e rodapés, portanto, envolve:
%
% 1. Redefinir \chaptermark e \sectionmark (ambas recebem um parâmetro,
%    que é o nome do capítulo/seção) para fazê-los executar \markboth e
%    \markright, inserindo o conteúdo desejado nas marks;
%
% 2. Usar \fancyhead e \fancyfoot para definir os cabeçalhos e rodapés
%    usando o conteúdo de \leftmark e \rightmark.
%
% Além das marks, é possível usar também as variáveis \thepage (número
% da página), \thechapter (número do capítulo) e \thesection (número da
% seção).
%
% Neste modelo, modificamos chaptermark e sectionmark para armazenar
% o número e nome do capítulo em leftmark e o número e nome da seção
% em rightmark. MAS imprimimos rightmark nas páginas pares (esquerda)
% e leftmark nas páginas ímpares (direita)!


%%%%%%%%%%

% Sem linha separando o cabeçalho
\renewcommand{\headrulewidth}{0pt}

% Queremos colocar o número da página mais próximo da borda do papel (na
% horizontal). Para isso, vamos aumentar \headwidth, somando "tamanho da
% margem direita -10mm" (o número vai ficar a 10mm da borda).
%
% Observe que a package geometry define \evensidemargin, mas seu valor não
% necessariamente corresponde ao que queremos aqui (não sei bem como nem
% por que geometry define esse valor). Ao invés de usá-lo, vamos calcular
% manualmente.
%
% A distância entre a borda esquerda/interna do papel e o início do texto
% é dada por (1in + \hoffset + \oddsidemargin); a margem direita, portanto,
% é dada por (\paperwidth - (1in + \hoffset + \oddsidemargin + \textwidth)).
\dimdef{\othermargin}{\paperwidth - 1in - \hoffset - \oddsidemargin - \textwidth}
\addtolength{\headwidth}{\othermargin}
\addtolength{\headwidth}{-10mm}

\newcommand{\formataNumPaginas}{
  \fancyhead[RO]{\raisebox{8mm}{\footnotesize\bfseries\thepage}}
  \fancyhead[LE]{\raisebox{8mm}{\footnotesize\bfseries\thepage}}
}

\newcommand{\formataCabecalhosDinamicos}{
  \fancyhead[LO]{\scriptsize\MakeTextUppercase{\rightmark}}
  \fancyhead[RE]{\scriptsize\MakeTextUppercase{\leftmark}}
}

\fancypagestyle{mainmatter}{
  % Número e nome do capítulo no cabeçalho das páginas pares
  % (e nas ímpares quando não há seções)
  \renewcommand{\chaptermark}[1]{
    \markboth
      {\thechapter\hskip 0.3em |\hskip 0.3em ##1}
      {\thechapter\hskip 0.3em |\hskip 0.3em ##1}
  }

  % Número e nome da seção no cabeçalho das páginas ímpares
  \renewcommand{\sectionmark}[1]{
    \markright{\thesection\hskip 0.3em |\hskip 0.3em ##1}
  }

  \fancyhf{}
  \formataNumPaginas{}
  \formataCabecalhosDinamicos{}
}

% Formato para capítulos não-numerados mas que devem
% aparecer no sumário (tipicamente, a introdução):
\fancypagestyle{onlynames}{
  % Só o nome do capítulo/seção, sem numeração
  \renewcommand{\chaptermark}[1]{\markboth{##1}{##1}}
  \renewcommand{\sectionmark}[1]{\markright{##1}}

  \fancyhf{}
  \formataNumPaginas{}
  \formataCabecalhosDinamicos{}
}

\fancypagestyle{appendix}{
  \renewcommand{\chaptermark}[1]{%
    \markboth{%
      % Páginas ímpares: "Apêndice/Anexo X"
      \@chapapp\ \thechapter%
    }{%
      % Páginas pares: "X | nome do apêndice/anexo"
      \thechapter\hskip 0.3em |\hskip 0.3em ##1%
    }
  }

  \fancyhf{}
  \formataNumPaginas{}
  \formataCabecalhosDinamicos{}
}

% Na frontmatter e backmatter, não há números de capítulos e (em geral)
% não há subdivisões (capítulos/seções/subseções), apenas um nível.
% O correto, então, é usar o mesmo texto (nome do capítulo ou seção,
% sem número) nas páginas pares ou ímpares.
%
% Normalmente, a bibliografia e o índice remissivo não definem os
% cabeçalhos (não executam "chaptermark/sectionmark"), mas "forçamos"
% isso manualmente (em bibconfig.tex e index.tex).
\fancypagestyle{backmatter}{
  \renewcommand{\chaptermark}[1]{\markboth{##1}{##1}}
  \renewcommand{\sectionmark}[1]{\markboth{##1}{##1}}

  \fancyhf{}
  \formataNumPaginas{}
  \formataCabecalhosDinamicos{}
}

% A página inicial de cada capítulo é automaticamente configurada para o estilo
% "plain", então vamos definir esse estilo (colocando o número de página no
% mesmo lugar das demais). As páginas iniciais também usam esse estilo.
\fancypagestyle{plain}{
  \fancyhf{}
  \formataNumPaginas{}
}


% biblatex pode ser configurado para inserir a bibliografia no sumário;
% bibtex não oferece essa possibilidade. Com esta package, resolvemos
% esse problema.
\usepackage[nottoc,notlot,notlof]{tocbibind}

% Só olha até o nível 2 (subseções) para gerar o sumário e os
% cabeçalhos, ou seja, não coloca nomes de subsubseções (nível 3)
% no sumário nem nos cabeçalhos.
\setcounter{tocdepth}{2}

% Só numera até o nível 2 (subseções, como 2.3.1), ou seja, não numera
% sub-subseções (como 2.3.1.1). Veja que isso afeta referências
% cruzadas: se você fizer \ref{uma-sub-subsecao} sem que ela seja
% numerada, a referência vai apontar para a seção um nível acima.
\setcounter{secnumdepth}{2}

% Normalmente, o capítulo de introdução não deve ser numerado, mas
% deve aparecer no sumário. Por padrão, LaTeX não oferece uma solução
% para isso, então criamos aqui os comandos \unnumberedchapter,
% \unnumberedsection e \unnumberedsubsection.
\newcommand{\unnumberedchapter}[2][]{
  \ifblank{#1}
    {
      \chapter*{#2}
      \phantomsection
      \addcontentsline{toc}{chapter}{#2}
      \chaptermark{#2}
    }
    {
      \chapter*{#2}
      \phantomsection
      \addcontentsline{toc}{chapter}{#1}
      \chaptermark{#1}
    }
}

\newcommand{\unnumberedsection}[2][]{
  \ifblank{#1}
    {
      \section*{#2}
      \phantomsection
      \addcontentsline{toc}{section}{#2}
      \sectionmark{#2}
    }
    {
      \section*{#2}
      \phantomsection
      \addcontentsline{toc}{section}{#1}
      \sectionmark{#1}
    }
}

\newcommand{\unnumberedsubsection}[2][]{
  \ifblank{#1}
    {
      \subsection*{#2}
      \phantomsection
      \addcontentsline{toc}{subsection}{#2}
    }
    {
      \subsection*{#2}
      \phantomsection
      \addcontentsline{toc}{subsection}{#1}
    }
}


%%%%%%%%%%%%%%%%%%%%%%%%%%%%%%%%%%%%%%%%%%%%%%%%%%%%%%%%%%%%%%%%%%%%%%%%%%%%%%%%
%%%%%%%%%%%%%%%%%%%%%%%%%% ESPAÇAMENTO E ALINHAMENTO %%%%%%%%%%%%%%%%%%%%%%%%%%%
%%%%%%%%%%%%%%%%%%%%%%%%%%%%%%%%%%%%%%%%%%%%%%%%%%%%%%%%%%%%%%%%%%%%%%%%%%%%%%%%

% LaTeX por default segue o estilo americano e não faz a indentação da
% primeira linha do primeiro parágrafo de uma seção; este pacote ativa essa
% indentação, como é o estilo brasileiro
\usepackage{indentfirst}

% A primeira linha de cada parágrafo costuma ter um pequeno recuo para
% tornar mais fácil visualizar onde cada parágrafo começa. Além disso, é
% possível colocar um espaço em branco entre um parágrafo e outro. Esta
% package coloca o espaço em branco e desabilita o recuo; como queremos
% o espaço *e* o recuo, é preciso guardar o valor padrão do recuo e
% redefini-lo depois de carregar a package.
% TODO: depois que ubuntu 18.04 se tornar obsoleta (abril/2023), remover
%       as linhas "oldparindent" e carregar a package com a opção "indent".
\newlength\oldparindent
\setlength\oldparindent\parindent
\usepackage[parfill]{parskip}
\setlength\parindent\oldparindent


%%%%%%%%%%%%%%%%%%%%%%%%%%%%%%%%%%%%%%%%%%%%%%%%%%%%%%%%%%%%%%%%%%%%%%%%%%%%%%%%
%%%%%%%%%%%%%%%%%%%%%%%%%% EPÍGRAFE E NOTAS DE RODAPÉ %%%%%%%%%%%%%%%%%%%%%%%%%%
%%%%%%%%%%%%%%%%%%%%%%%%%%%%%%%%%%%%%%%%%%%%%%%%%%%%%%%%%%%%%%%%%%%%%%%%%%%%%%%%

% O formato padrão do pacote epigraph é bem feinho...
% Outra opção para epígrafes é o pacote quotchap
\usepackage{epigraph}

\setlength\epigraphwidth{.85\textwidth}

% Sem linha entre o texto e o autor
\setlength{\epigraphrule}{0pt}

% Ambiente auxiliar para colocar margem à direita da epígrafe
% (como sempre, o modo mais simples de mudar as margens de um
% pagrágrafo é fazer de conta que é uma lista)
\newenvironment{epShiftLeft}
  {
    \par\begin{list}{}
      {
        \leftmargin 0pt
        \labelwidth 0pt
        \labelsep 0pt
        \itemsep 0pt
        \topsep 0pt
        \partopsep 0pt
        \rightmargin 2em
      }
    \item\FlushRight
  }
  {
    \end{list}
    % O espaço padrão que epigraph coloca entre a citação
    % e o autor é muito pequeno; vamos aumentar um pouco
    \vspace*{.3\baselineskip}
  }

\renewcommand\textflush{epShiftLeft}
\renewcommand\sourceflush{epShiftLeft}

\newcommand{\epigrafe}[2] {%
  \ifstrempty{#2}{
    \epigraph{\itshape #1}{}
  }{
    \epigraph{\itshape #1}{--- #2}
  }
}

% Formato personalizado para as notas de rodapé. Copiado quase
% literalmente do exemplo na documentação das classes-padrão de
% LaTeX2e (texdoc classes). Seria possível fazer algo similar
% usando list{} com um único item usando \@thefnmark como label.

% \footnotesep não é um espaço adicional, mas sim um strut que
% existe no começo de cada nota. É por isso que o valor é "grande"
% (\baselineskip) mas a separação de fato é pequena.

\renewcommand\@makefntext[1]{%
    \setlength{\footnotesep}{1\baselineskip}%
    \@setpar{%
        \@@par
        \@tempdima = \hsize
        \advance\@tempdima-4pt\relax
        \parshape \@ne 4pt \@tempdima
    }%
    \par
    \parindent 1em\noindent
    \parskip .3\baselineskip
    \hbox to \z@{\hss\@makefnmark\,}#1%
}

% \maketitle redefine as notas de rodapé (\thanks) para usar símbolos
% ao invés de números, mas essa não é a única mudança. \maketitle
% também muda \@makefnmark para que a indicação de nota de rodapé
% não ocupe espaço horizontal (isso é feito com \rlap). Isso é feito
% porque a lista de autores em geral é similar a
% \author{Fulano\thanks{instituição 1}, Ciclano\thanks{instituição 2}}.
% Com essa mudança, a nota aparece acima da vírgula entre os autores.
% Mas isso significa que\maketitle precisa também modificar \@makefntext
% para que esse efeito aconteça apenas na lista de autores e não na
% nota em si. Assim, como criamos um novo formato para as notas de
% rodapé, precisamos mudar o formato em \maketitle também.

\newcommand\@maketitlemakefntext[1]{%
    \setlength{\footnotesep}{1\baselineskip}%
    \@setpar{%
        \@@par
        \@tempdima = \hsize
        \advance\@tempdima-4pt\relax
        \parshape \@ne 4pt \@tempdima
    }%
    \par
    \parindent 1em\noindent
    \parskip .3\baselineskip
    \hbox to \z@{\hss\@textsuperscript{\normalfont\@thefnmark}\,}#1%
}

\patchcmd\maketitle
  {\long\def\@makefntext}
  {\let\@makefntext\@maketitlemakefntext\long\def\@disabledmakefntext}
  {}{}

% Algumas packages mais novas que tratam de fontes funcionam corretamente
% tanto com fontspec (LuaLaTeX/XeLaTeX) quanto com NFSS (qualquer versão
% de LaTeX, mas menos poderoso que fontspec). No entanto, muitas funcionam
% apenas com NFSS. Nesse caso, em LuaLaTeX/XeLaTeX é melhor usar os
% comandos de fontspec, como exemplificado mais abaixo.

% É possível mudar apenas uma das fontes. Em particular, a fonte
% teletype da família Computer Modern foi criada para simular
% as impressoras dos anos 1970/1980. Sendo assim, ela é uma fonte (1)
% com serifas e (2) de espaçamento fixo. Hoje em dia, é mais comum usar
% fontes sem serifa para representar código-fonte. Além disso, ao imprimir,
% é comum adotar fontes que não são de espaçamento fixo para fazer caber
% mais caracteres em uma linha de texto. Algumas opções de fontes para
% esse fim:
%\usepackage{newtxtt} % Não funciona com fontspec (lualatex / xelatex)
%\usepackage{DejaVuSansMono}
% inconsolata é uma boa fonte, mas não tem variante itálico
%\ifPDFTeX
%  \usepackage[narrow]{inconsolata}
%\else
%  \setmonofont{inconsolatan}
%\fi
\usepackage[scale=.85]{sourcecodepro}

% Ao invés da família Computer Modern, é possível usar outras como padrão.
% Uma ótima opção é a libertine, similar (mas não igual) à Times mas com
% suporte a Small Caps e outras qualidades. A fonte teletype da família
% é serifada, então é melhor definir outra; a opção "mono=false" faz
% o pacote não carregar sua própria fonte, mantendo a escolha anterior.
% Versões mais novas de LaTeX oferecem um fork desta fonte, libertinus.
% As packages libertine/libertinus funcionam corretamente com pdfLaTeX,
% LuaLaTeX e XeLaTeX.
% TODO: remover suporte a Libertine no final de 2022

\IfFileExists{libertinus.sty}
    {
      \usepackage[mono=false]{libertinus}
      % Com LuaLaTeX/XeLaTeX, Libertinus configura também
      % a fonte matemática; aqui só precisamos corrigir \mathit
      \ifLuaTeX
        \setmathfontface\mathit{Libertinus Serif Italic}
      \fi
      \ifXeTeX
        % O nome de arquivo da fonte mudou na versão 2019-04-04
        \@ifpackagelater{libertinus-otf}{2019/04/03}
            {\setmathfontface\mathit{LibertinusSerif-Italic.otf}}
            {\setmathfontface\mathit{libertinusserif-italic.otf}}
      \fi
    }
    {
      % Libertinus não está disponível; vamos usar libertine
      \usepackage[mono=false]{libertine}

      % Com Libertine, é preciso modificar também a fonte
      % matemática, além de \mathit
      \ifLuaTeX
        \setmathfont{Libertinus Math}
        \setmathfontface\mathit{Linux Libertine O Italic}
      \fi

      \ifXeTeX
        \setmathfont{libertinusmath-regular.otf}
        \setmathfontface\mathit{LinLibertine_RI.otf}
      \fi
    }

\ifPDFTeX
  % A família libertine por padrão não define uma fonte matemática
  % específica para pdfLaTeX; uma opção que funciona bem com ela:
  %\usepackage[libertine]{newtxmath}
  % Outra, provavelmente melhor:
  \usepackage{libertinust1math}
\fi

% Ativa apenas a fonte biolinum, que é a fonte sem serifa da família.
%\IfFileExists{libertinus.sty}
%  \usepackage[sans]{libertinus}
%\else
%  \usepackage{biolinum}
%\fi

% Também é possível usar a Times como padrão; nesse caso, a fonte
% sem serifa usualmente é a Helvetica. Mas provavelmente libertine
% é uma opção melhor.
%\ifPDFTeX
%  \usepackage[helvratio=0.95,largesc]{newtxtext}
%  \usepackage{newtxtt} % Fonte teletype
%  \usepackage{newtxmath}
%\else
%  % Clone da fonte Times como fonte principal
%  \setmainfont{TeX Gyre Termes}
%  \setmathfont[Scale=MatchLowercase]{TeX Gyre Termes Math}
%  % TeX Gyre Termes Math tem um bug e não define o caracter
%  % \setminus; Vamos contornar esse problema usando apenas
%  % esse caracter da fonte STIX Two Math
%  \setmathfont[range=\setminus]{STIX Two Math}
%  % Clone da fonte Helvetica como fonte sem serifa
%  \setsansfont{TeX Gyre Heros}
%  % Clone da Courier como fonte teletype, mas provavelmente
%  % é melhor utilizar sourcecodepro
%  %\setmonofont{TeX Gyre Cursor}
%\fi

% Cochineal é outra opção de qualidade; ela define apenas a fonte
% com serifa.
%
% Com NFSS (recomendado no caso de cochineal):
%\usepackage{cochineal}
%\usepackage[cochineal,vvarbb]{newtxmath}
%\usepackage[cal=boondoxo]{mathalfa}
%
% Com fontspec (até a linha "setmathfontface..."):
%
%\setmainfont{Cochineal}[
%  Extension=.otf,
%  UprightFont=*-Roman,
%  ItalicFont=*-Italic,
%  BoldFont=*-Bold,
%  BoldItalicFont=*-BoldItalic,
%  %Numbers={Proportional,OldStyle},
%]
%
%\DeclareRobustCommand{\lfstyle}{\addfontfeatures{Numbers=Lining}}
%\DeclareTextFontCommand{\textlf}{\lfstyle}
%\DeclareRobustCommand{\tlfstyle}{\addfontfeatures{Numbers={Tabular,Lining}}}
%\DeclareTextFontCommand{\texttlf}{\tlfstyle}
%
%% Cochineal não tem uma fonte matemática; com fontspec, provavelmente
%% o melhor a fazer é usar libertinus.
%\setmathfont{Libertinus Math}
%\setmathfontface\mathit{Cochineal-Italic.otf}

% gentium inclui apenas uma fonte serifada, similar a Garamond, que busca
% cobrir todos os caracteres unicode
%\usepackage{gentium}

% LaTeX normalmente funciona com fontes que foram adaptadas para ele, ou
% seja, ele não usa as fontes padrão instaladas no sistema: para usar
% uma fonte é preciso ativar o pacote correspondente, como visto acima.
% É possível escapar dessa limitação e acessar as fontes padrão do sistema
% com XeTeX ou LuaTeX. Com eles, além dos pacotes de fontes "tradicionais",
% pode-se usar o pacote fontspec para usar fontes do sistema.
%\usepackage{fontspec}
%\setmainfont{DejaVu Serif}
%\setmainfont{Charis SIL}
%\setsansfont{DejaVu Sans}
%\setsansfont{Libertinus Sans}[Scale=1.1]
%\setmonofont{DejaVu Sans Mono}

% fontspec oferece vários recursos interessantes para manipular fontes.
% Por exemplo, Garamond é uma fonte clássica; a versão EBGaramond é muito
% boa, mas não possui versões bold e bold-italic; aqui, usamos
% CormorantGaramond ou Gentium para simular a versão bold.
%\setmainfont{EBGaramond12}[
%  Numbers        = {Lining,} ,
%  Scale          = MatchLowercase ,
%  UprightFont    = *-Regular ,
%  ItalicFont     = *-Italic ,
%  BoldFont       = gentiumbasic-bold ,
%  BoldItalicFont = gentiumbasic-bolditalic ,
%%  BoldFont       = CormorantGaramond Bold ,
%%  BoldItalicFont = CormorantGaramond Bold Italic ,
%]
%
%\newfontfamily\garamond{EBGaramond12}[
%  Numbers        = {Lining,} ,
%  Scale          = MatchLowercase ,
%  UprightFont    = *-Regular ,
%  ItalicFont     = *-Italic ,
%  BoldFont       = gentiumbasic-bold ,
%  BoldItalicFont = gentiumbasic-bolditalic ,
%%  BoldFont       = CormorantGaramond Bold ,
%%  BoldItalicFont = CormorantGaramond Bold Italic ,
%]

% Crimson tem Small Caps, mas o recurso é considerado "em construção".
% Vamos utilizar Gentium para Small Caps
%\setmainfont{Crimson}[
%  Numbers           = {Lining,} ,
%  Scale             = MatchLowercase ,
%  UprightFont       = *-Roman ,
%  ItalicFont        = *-Italic ,
%  BoldFont          = *-Bold ,
%  BoldItalicFont    = *-Bold Italic ,
%  SmallCapsFont     = Gentium Plus ,
%  SmallCapsFeatures = {Letters=SmallCaps} ,
%]
%
%\newfontfamily\crimson{Crimson}[
%  Numbers           = {Lining,} ,
%  Scale             = MatchLowercase ,
%  UprightFont       = *-Roman ,
%  ItalicFont        = *-Italic ,
%  BoldFont          = *-Bold ,
%  BoldItalicFont    = *-Bold Italic ,
%  SmallCapsFont     = Gentium Plus ,
%  SmallCapsFeatures = {Letters=SmallCaps} ,
%]

% Com o pacote fontspec, também é possível usar o comando "\fontspec" para
% selecionar uma fonte temporariamente, sem alterar as fontes-padrão do
% documento.

% Captions com fonte menor, indentação normal, corpo do texto
% negrito e nome do caption itálico
\usepackage[
  font=small,
  format=plain,
  labelfont={bf,up},
  textfont={normalfont,it}]{caption}

% Em geral, a package caption é capaz de "adivinhar" se o caption
% está acima ou abaixo da figura/tabela, mas isso não funciona
% corretamente com longtable. Aqui, forçamos a package a considerar
% que os captions ficam abaixo das tabelas.
\captionsetup*[longtable]{position=bottom}

\makeatother
