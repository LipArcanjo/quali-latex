%%%%%%%%%%%%%%%%%%%%%%%%%%%%%%%%%%%%%%%%%%%%%%%%%%%%%%%%%%%%%%%%%%%%%%%%%%%%%%%%
%%%%%%%%%%%%%%%%%%%%%%%%%%%%% FIGURAS / FLOATS %%%%%%%%%%%%%%%%%%%%%%%%%%%%%%%%%
%%%%%%%%%%%%%%%%%%%%%%%%%%%%%%%%%%%%%%%%%%%%%%%%%%%%%%%%%%%%%%%%%%%%%%%%%%%%%%%%

% LaTeX escolhe automaticamente o "melhor" lugar para colocar cada float.
% Por padrão, ele tenta colocá-los no topo da página e depois no pé da
% página; se não tiver sucesso, vai para a página seguinte e recomeça.
% Se esse algoritmo não tiver sucesso "logo", LaTeX cria uma página só
% com floats. É possível modificar esse comportamento com as opções de
% posicionamento: "tp", por exemplo, instrui LaTeX a considerar apenas
% o topo da página ou uma página só de floats (ignorando o pé da página),
% e "htbp" o instrui para tentar "aqui" como a primeira opção. A ordem
% dessas opções não é relevante: dentre as opções disponíveis, LaTeX
% sempre tenta "aqui, topo, pé, página". Os pacotes "float" e "floatrow"
% acrescentam a opção "H", que significa "aqui, incondicionalmente".
%
% A escolha do "melhor" lugar leva em conta os parâmetros abaixo, mas é
% possível ignorá-los com a opção de posicionamento "!". Dado que os
% valores default não são muito bons para floats "grandes" ou documentos
% com muitos floats, é muito comum usar "!" ou "H". No entanto, modificando
% estes parâmetros o algoritmo automático tende a funcionar melhor. Ainda
% assim, vale ler a discussão a respeito na seção "Limitações do LaTeX"
% deste modelo.

% Fração da página que pode ser ocupada por floats no topo. Default: 0.7
\renewcommand{\topfraction}{.8}
% Idem para documentos em colunas e floats que tomam as 2 colunas. Default: 0.7
%\renewcommand{\dbltopfraction}{.7}
% Fração da página que pode ser ocupada por floats no pé. Default: 0.3
%\renewcommand{\bottomfraction}{.3}
% Fração mínima da página que deve conter texto. Default: 0.2
%\renewcommand{\textfraction}{.2}
% Numa página só de floats, fração mínima que deve ser ocupada. Default: 0.5
% floatpagefraction *deve* ser menor que topfraction.
\renewcommand{\floatpagefraction}{.66}
% Idem para documentos em colunas e floats que tomam as 2 colunas. Default: 0.5
\renewcommand{\dblfloatpagefraction}{.66}
% Máximo de floats no topo da página. Default: 2
\setcounter{topnumber}{3}
% Idem para documentos em colunas e floats que tomam as 2 colunas. Default: 2
%\setcounter{dbltopnumber}{2}
% Máximo de floats no pé da página. Default: 1
\setcounter{bottomnumber}{2}
% Máximo de floats por página. Default: 3
\setcounter{totalnumber}{5}

% A package float é amplamente utilizada; ela permite definir novos tipos
% de float e também acrescenta a possibilidade de definir "H" como opção de
% posicionamento do float, que significa "aqui, incondicionalmente". No
% entanto, ela tem algumas fragilidades e não é atualizada desde 2001.
% floatrow é uma versão aprimorada e com mais recursos da package "float",
% mas também não é atualizada desde 2009. Aqui utilizamos alguns recursos
% disponibilizados por ambas e é possível escolher qual delas utilizar.
% Um dos principais recursos dessas packages é permitir a criação de novos
% tipos de float; veja o arquivo source-code.tex para um exemplo.
%\usepackage{float}
\usepackage{floatrow}

% Por padrão, LaTeX prefere colocar floats no topo da página que
% onde eles foram definidos; vamos mudar isso. Este comando depende
% do pacote "floatrow", carregado logo acima.
\floatplacement{table}{htbp}
\floatplacement{figure}{htbp}

% Em alguns casos, um float pode aparecer antes do local do texto em que
% foi definido (ou seja, no topo da página ao invés do meio da página).
% Esta package garante que floats (tabelas e figuras) só apareçam após
% o local no texto em que foram definidos; veja os detalhes em
% https://tex.stackexchange.com/a/297580 . Note que, se o float tem a
% opção "h", normalmente LaTeX *não* coloca o float no topo da página
% atual: se o float não pode ser colocado "here", ele é delegado para
% a página seguinte. Portanto, com a opção "h" flafter em geral não faz
% diferença.
\usepackage{flafter}

% Em documentos com duas colunas, floats normalmente são colocados como
% parte de uma das colunas. No entanto, é possível usar "\begin{figure*}"
% ou "\begin{table*}" para criar floats que ocupam as duas colunas. Floats
% "duplos" desse tipo têm algumas limitações:
%
% 1. Mesmo que haja espaço disponível na página atual, eles são sempre
%    inseridos na página seguinte ao lugar em que foram definidos (então
%    é comum defini-los antes do lugar "certo" para compensar isso)
%
% 2. Eles só podem aparecer no topo da página ou em uma página de floats,
%    ou seja, nunca "here" nem no pé da página.
%
% 3. Em alguns casos, eles podem aparecer fora da ordem em relação aos
%    demais floats do mesmo tipo (o que não acontece com floats "normais")
%
% Esta package:
%
% 1. Soluciona parcialmente o primeiro problema: floats "duplos" podem
%    aparecer na página em que são definidos se sua definição está contida
%    no texto da coluna da esquerda;
%
% 2. Soluciona o segundo problema: floats "duplos" podem aparecer tanto no
%    topo quanto no pé da página. Observe que eles *não* podem aparecer
%    "here" porque isso não faz sentido: a figura interromperia o fluxo
%    do texto da "outra" coluna (ainda assim, as packages midfloat e cuted
%    permitem fazer isso).
%
% 3. Soluciona o terceiro problema.
\usepackage{stfloats}

% Captions com fonte menor, indentação normal, corpo do texto
% negrito e nome do caption itálico
\usepackage[
  font=small,
  format=plain,
  labelfont={bf,up},
  textfont={normalfont,it}]{caption}

% Em geral, a package caption é capaz de "adivinhar" se o caption
% está acima ou abaixo da figura/tabela, mas isso não funciona
% corretamente com longtable. Aqui, forçamos a package a considerar
% que os captions ficam abaixo das tabelas.
\captionsetup*[longtable]{position=bottom}

% Permite importar figuras. LaTeX "tradicional" só é capaz de trabalhar com
% figuras EPS; hoje em dia não há nenhuma boa razão para usar essa versão.
% Já pdfTeX, XeTeX e LuaTeX podem usar figuras nos formatos PDF, JPG e PNG.
% Em algumas instalações, essas versões conseguem converter automaticamente
% arquivos EPS para PDF, mas não isso é garantido, então é melhor evitar o
% formato EPS.
\usepackage{graphicx}
