\en

\section{Final Remarks}

We have completed Phase 1 of our research, during which we developed our proposed tool and evaluated it using the BigCloneBench dataset and through empirical analyses. The results indicate our tool can identify code clone duplications within the Linux kernel.

As the next step, we plan to conduct an ethnographic study as outlined in Section \ref{sec:meteth}. To preliminarily assess the potential impact of this research, we applied our systematic approach for mitigating code duplication to two pairs of functions. Table \ref{tab:patch} presents the initial results obtained from this process.

\begin{table}
\begin{tabular}{ | c | c | c | c | m{6em} | }

\hline

\textbf{Mitigation} & \textbf{Files changed} & \textbf{Lines added} & \textbf{Lines removed} & \textbf{Refactoring methods used}
\\ \hline 

1 & 13 & +224 & -753 & Parameterize Method, Extract Method  \\ \hline
2 & 8 & +132 & -517 & None \\ \hline

\hline
\end{tabular}
\caption{Early results on mitigating code duplications on the AMD Display driver.}
\label{tab:patch}
\end{table}


In the first function pair, completing the refactoring requires the expertise of a maintainer for the AMD Display driver. This observation is because duplicated code artifacts involve copy-pasted code that imports different configuration files. The design of the AMD Display driver configuration files necessitates importing specific configurations at compile time, which prevents a generic approach. Consequently, refactoring this pair requires changes in how the AMD Display driver manages configuration files -- a complex task that demands input from an expert or maintainer.

The refactoring of the second function pair, however, was successful. We submitted this refactoring as a patch to the AMD Display driver to obtain feedback from the maintainers.
\todo[inline]{Atualizar ... tentar colocar algo do feedback que recebeu da revisão do patch}


%%%%%%% Cronograma %%%%%%%

\begin{figure}
  \centering

  % Package pgfgantt; veja o arquivo imegoodies.sty, em que vários
  % aspectos da aparência deste diagrama foram definidos.
  \begin{ganttchart}[
                     time slot format=isodate-yearmonth,
                     time slot unit=month,
                    ]{2024-10}{2025-5}

    \gantttitlecalendar{year,month=shortname} \ganttnewline
	\ganttbar{
		Sistematic approach to \ganttalignnewline
		mitigate code duplication
	}{2024-10}{2025-2} \ganttnewline

	\ganttbar{
		Submit mitigations to \ganttalignnewline
		the Linux mailing list  
	}{2024-11}{2025-3} \ganttnewline
 	\ganttbar{Analysis results }{2025-4}{2025-4} \ganttnewline
 	\ganttbar{Thesis Writing }{2025-2}{2025-5} \ganttnewline
	\ganttmilestone{Thesis Submission}{2025-5}

   \end{ganttchart}

  \caption{Exemplo de cronograma.\label{fig:cronograma}}
\end{figure}

\todo[inline]{Atualizar o cronograma, alocando tempo para evolucao da ferramenta}

Finally, as indicated in our proposed schedule in Table \ref{fig:cronograma}, our upcoming tasks include conducting the complete ethnographic study. This step will start with completing our systematic approach to code duplication mitigation. Once we have finalized the mitigations, we intend to submit them to the AMD Display driver mailing list to gather insights and feedback from the maintainers regarding the changes we implemented.

\todo[inline]{colocar aqui também algo da evolucao da ferramenta que sera feito até o final do mestrado}

