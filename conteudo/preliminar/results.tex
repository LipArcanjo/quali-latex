\en

\section{Final Remarks}


We have completed Phase 1 of our research, during which we developed our proposed tool and evaluated it using the BigCloneBench dataset and through empirical analyses. The results indicate our tool can identify code clone duplications within the Linux kernel.

As the next step, we plan to conduct an ethnographic study as outlined in Section \ref{sec:meteth}. To preliminarily assess the potential impact of this research, we applied our systematic approach for mitigating code duplication to two pairs of functions. Table \ref{tab:patch} presents the initial results obtained from this process.

\begin{table}
\begin{tabular}{ | c | c | c | c | m{6em} | }

\hline

\textbf{Mitigation} & \textbf{Files changed} & \textbf{Lines added} & \textbf{Lines removed} & \textbf{Refactoring methods used}
\\ \hline 

1 & 13 & +224 & -753 & Parameterize Method, Extract Method  \\ \hline
2 & 8 & +132 & -517 & None \\ \hline

\hline
\end{tabular}
\caption{Early results on mitigating code duplications on the AMD Display driver.}
\label{tab:patch}
\end{table}


In the first function pair, completing the refactoring requires the expertise of a maintainer for the AMD Display driver. This observation is because duplicated code artifacts involve copy-pasted code that imports different configuration files. The design of the AMD Display driver configuration files necessitates importing specific configurations at compile time, which prevents a generic approach. Consequently, refactoring this pair requires changes in how the AMD Display driver manages configuration files -- a complex task that demands input from an expert or maintainer.

The refactoring of the second function pair, however, was successful. We submitted this refactoring as a patch to the AMD Display driver to obtain feedback from the maintainers. The process of sending the patch took more time than expected. We initially submitted the patch on August 9th, 2024, but we did not receive replies. As a result, we had to resend it on October 9th, 2024 and received an initial response on November 3rd, 2024. 

We understood new details about the driver evolution processes in discussions with our partner from the AMD Display driver. Some of the code in the driver is shared across all operating systems that support the implemented GPUs. This fact creates a complex process within AMD to format and submit changes across the supported systems while simultaneously implementing measures to mitigate errors.

Additionally, we comprehended that AMD Display driver developers can not view duplicated code negatively, as we understand it is a good practice in software engineering. One example shared with us is that duplicated code enhances the independence of GPU driver code, allowing developers to make changes to a specific GPU without needing to test compatibility with others. This approach helps save significant time and effort. 

Regarding the patch content, only minor changes were requested to align with code 
style and best practices that were initially unknown to us. More details of the 
patch can be found in Appendix \ref{app:feedback}.


%%%%%%% Cronograma %%%%%%%

\begin{figure}
  \centering

  % Package pgfgantt; veja o arquivo imegoodies.sty, em que vários
  % aspectos da aparência deste diagrama foram definidos.
  \begin{ganttchart}[
                     time slot format=isodate-yearmonth,
                     time slot unit=month,
                    ]{2024-10}{2025-5}

    \gantttitlecalendar{year,month=shortname} \ganttnewline
	\ganttbar{
		Sistematic approach to \ganttalignnewline
		mitigate code duplication
	}{2024-10}{2025-2} \ganttnewline

	\ganttbar{
		Submit mitigations to \ganttalignnewline
		the Linux mailing list  
	}{2024-11}{2025-3} \ganttnewline
 	\ganttbar{Analysis results }{2025-4}{2025-4} \ganttnewline
 	\ganttbar{Thesis Writing }{2025-2}{2025-5} \ganttnewline
	\ganttmilestone{Thesis Submission}{2025-5}

   \end{ganttchart}

  \caption{Exemplo de cronograma.\label{fig:cronograma}}
\end{figure}


Finally, as indicated in our proposed schedule in Table \ref{fig:cronograma}, 
our upcoming tasks include conducting the complete ethnographic study.
This step will start with completing our systematic approach to code 
duplication mitigation. Once we have finalized the mitigations, we intend 
to submit them to the AMD Display driver mailing list to gather insights 
and feedback from the maintainers regarding the changes we implemented.

Our upcoming tasks also include implementing improvements to our proposed tool. 
While the tool’s current state allowed us to conduct the experiments described in 
Chapter \ref{cha:method},
it is not yet adequately user-friendly or manageable for external users. 
To achieve our complementary goal of providing a solution for code duplication 
in the kernel, we plan to improve the documentation, refine the code structure, 
and address any issues we encounter along the way, providing the Linux community 
with a way to tackle the issue.
