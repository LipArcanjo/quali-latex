\en

We finished Phase 1 of our research, where we developed our proposed tool and 
evaluated the tool on the BigCloneBench dataset and with our empirical analyses. 
We discussed the results found to understand that the tool is capable of identifying 
code clone duplications in the Linux kernel. With these results, we answered RQ1.1 
(\textit{How accurate is the proposed tool for detecting duplicate functions?}).

Related to our RQ1
(\textit{How effective is our approach in mitigating duplicate code in the Linux kernel?}),
we intend to approach our ethnographic study presented in \ref{sec:meteth} as the next steps 
in our work. As an early validation to see the potential of this research, we conducted the 
proposed systematic approach to mitigate code duplication on two function pairs. Table
(CREATE TABLE AND ADD HERE) shows the early results found.

CREATE TABLE AND ADD HERE

The first function pair requires an expert/maintainer on the AMD Display driver to finalize 
the refactorization. The reason for this is that the duplicated code artifacts use 
copy-pasted code but import different configuration files. The AMD Display driver 
configurations files design requires the code that uses it to import the exact configuration 
in the compiler time without being generic. For this reason, the refactorization requires 
a change in the form the AMD Display driver deals with configuration files, which is a more 
complex task that requires an expert/maintainer on the AMD Display driver.

The second function pair refactorization was successful. At the moment of writing this 
report, we are sending the refactorization as a patch to the AMD Display driver to know 
the opinion of the maintainers of the driver.

\section{Next steps}

As shown in our proposed schedule (Figure CREATE FIGURE AND ADD HERE), as we advance, 
we intend to do the proper ethnographic study, starting by completing our systematic 
approach to mitigate code duplications, and with this mitigation on hands we intend 
to process to send them to the AMD Display driver mailing list to understand the 
AMD Display driver maintainers about the mitigations made by us.
