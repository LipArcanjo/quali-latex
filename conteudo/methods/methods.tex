\en

To answer our proposed research questions, we opted to do a multiple-method research 
where we covered the research question with different methodologies. To approach the RQ1, 
we opted to create an ethnographic study using participant-observation methodology where 
we plan to mitigate a selected subset of duplicated functions by refactoring them using 
a designed systematic approach while interacting with the Linux community to collect 
feedback on the mitigation made by us.


To answer RQ1, which covers our main objective to mitigate code duplication in the Linux 
kernel, we are first required to be able to identify them. For this reason, we developed 
a tool and need to understand if the tool can identify code duplications in the Linux kernel, 
which covers the supplementary research question RQ1.1 for the RQ1.

To answer RQ1.1, we applied two independent submethods and performed a triangulation of 
the results. The first submethod is to evaluate the proposed tool similarly to the literature, 
by comparing it against the BigCloneBench dataset \citep{bigclonebench}. The second submethod
for answering RQ1.1 is to do an empirical analysis on a selected subset of functions of the 
AMD Display driver that our tool infers is a duplication. With the two submethods completed, 
we will do a triangulation of results analyzing the findings. We opted for the triangulation 
of the two submethods because the formal literature method alone is limited and not fully 
trustworthy \citep{bigfail, litreview}.

\section{Evaluation of the tool by the formal literature}

\label{sec:metbig}

The BigCloneBench dataset \citep{bigclonebench} is a collection of Java code with a selection of
code duplication pairs categorized by the authors. These code duplication pairs are labeled by 
the code clone type described in \ref{subsec:types}.

Liut et al. presented a sample method to evaluate their code duplication detection method, 
in which they sample 20,000 pairs from each of the code duplication types with the addition 
of 20,000 nonduplication pairs to act as the negative samples \citep{tailor}. 
For a fair evaluation, we use the same sample method as them to evaluate our tool.

The proposed tool does not differentiate the code duplication types and infers duplications 
at the function level instead of the code file level, which is the standard in state-of-the-art 
methods. We were required to adjust the form we computed metrics of the evaluation. To accomplish 
this, we will define that every pair of functions in which the similarity metric is equal or 
superior to a threshold X will be considered a duplication, and the pair of code files in which 
the functions exist are labeled as duplication. If, on this method, a code duplication pair is 
correctly identified, it will count as correct to its code duplication types. If an incorrect 
duplication pair is inferred as a duplication, it will count as incorrect to all code duplication 
types. Given these definitions, we will compute the Recall metric presented in \ref{subsec:codemethods}
for each code duplication type.

To understand the variation of meaning when we change the similarity threshold, we will evaluate 
our tool for the given variations of the threshold X and discuss the results: 
30\%, 40\%, 50\%, 60\%, 70\%, 80\%, 90\%, 100\%.

\section{Evaluation of the tool by Empirical analyses}

\label{sec:metemp}

Given the problems of the BigCLoneBench presented by Krinke and Ragkhitwetsagul \citep{bigfail}
and the limitations of it in our scope, such as being a Java dataset while the Linux kernel 
is predominantly built in the C programming language, we opted to, together with the formal 
literature evaluation, perform an empirical study analyzing the codes in which our tool inferred 
as duplications.

The number of code lines and components in the Linux kernel makes it impractical to analyze every 
code duplication. To limit our scope, we will analyze solely code duplications in the AMD Display 
driver. This limitation still does not reduce the code duplication to analyze, in our preliminary 
tool execution shown in Figure \ref{fig:relatory_ex}, there are over 20,000 lines of code 
duplications on the AMD Display driver with a similarity metric of 100\%.

For our empirical analyses, we opted to make a random selection in the code duplication pairs 
inferred by our tool. To do that, we will use the definition that if the similarity metric of a 
function pair is greater or equal to X is considered a duplication. To understand the optimal 
values of X, we will analyze 10 functions pair random selected that have similarity metric X 
for each of the variations of the parameter X, which are 
30\%, 40\%, 50\%, 60\%, 70\%, 80\%, 90\%, 100\%. 
As it is unexpected to have code duplications with similarity equals a whole number, we will 
allow a variation of 1\% for greater or less to each of the variations of the parameter X.

To make the random selection of function pairs, we list the code duplications for each of the 
X variations and execute a random selection algorithm built into the source code language of 
our tool. The random selected function selected function pairs can 
be found in the Appendix (CREATE Appendix AND ADD HERE).

When we analyze the selected function pairs, we will empirically judge if they are duplications. 
We intend to collect the data on this empirical analysis and compare it to the formal literature 
method to understand if our tool detects duplicated functions and the difference it makes when 
varying the minimum similarity parameter of our tool.

\section{Ethnographic study}

\label{sec:meteth}

Ethnography is a research method recognized as a significant qualitative empirical approach to 
understanding people, their cultures, and their associated social and work practices \citep{bookethno}.
The central tenet of the ethnography approach is to enable researchers to understand values, 
beliefs, or ideas shared by a community under study from the members’ point of view 
\citep{ethnosoft}. Thus, the ethnographer acts as a community member, learning and observing 
what people do and why they do it.

Sharp et al. discuss the application of the ethnography method in software engineering research, 
demonstrating the low adoption of the method and its advantages in uncovering what practitioners 
do and why they do it \citep{ethnosoft}. In our research context, if we mitigate code 
duplications by ourselves without interacting with the Linux community, we cannot understand if 
the Linux maintainers would accept our approach for code duplication and what concerns they 
would have. Thus, we resorted to ethnography research, where we will act as a member of the 
Linux kernel and act as a contributor with a focus on removing code duplications found.

In this section, we will outline a systematic approach to mitigate code duplication in the 
Linux kernel. Next, we will detail the methodology for selecting a subset of duplicated functions 
identified by our tool. Finally, we will describe the ethnographic methodology for interacting 
with the Linux community.

\subsection{Systematic approach to mitigate code duplication}

\label{subsec:pipeline}

The C programming language’s way to communicate between source code files is through creating 
header files, which is a library specification \citep{Cbook}. As the Linux kernel 
is predominantly built in C, we remove code duplications by creating a library where the code 
artifacts will use the library instead of their duplications.

When removing duplicated functions, our desired behavior is to remove the functions in every 
portion of the code in which the functions exist, only allowing a unique version to exist in 
the library created. For this reason, given our initial duplicated function pair, we will use 
the function information functionality(Described in \ref{subsec:functioncommand}) to find all
places where this function exists in the source code. Thus, we finish with a collection of 
duplicated functions and the code files in which these functions exist.

Given the semantics of a library containing the shared code by a collection of code files, 
we added a step of the approach, which is searching for other functions contained in all of 
the code files of the collection we are working with. Thus, we have a collection of code files 
and a non-empty set of functions shared by all the code files in the collection.

Finally, for each of the functions shared by all code files, we make a specific-case approach 
to mitigate each one of them. If a function is shared by all the code files without 
modifications, we use a simple approach to only remove them and create the function in the 
source code file. If it is shared with modifications, we look for refactoring methods applicable 
to this specific case.

Finishing our mitigation method, if the mitigation ends up working in an initial step, we may 
move it forward to the Ethnographic method, interacting with the Linux community to understand 
their thoughts on the mitigation. If the mitigation ends up not working as expected, we will 
categorize it in two categories, the first one is that the refactoring requires a decision in 
which we, as not specialists and non-maintainers of the Linux kernel, cannot make it correctly,
although we believe a specialist would be able to refactor it. The second categorization is 
that we do not know if it is possible to refactor the duplication. With or without successful 
mitigation, we will store our mitigations on (ADD REP AND PUT HERE) to research transparency.

\subsection{Selection of the duplicated functions}

As discussed in the empirical analyses for the tool, it is impractical to evaluate every code 
duplication on the Linux kernel. Thus, similarly to the empirical analyses for the tool, we will 
limit the experiment scope to only the AMD Display driver and select a random sample of duplicated 
functions inferred by our tool to initiate the mitigation approach defined in \ref{subsec:pipeline}. 
We chose a sample size of 30 duplicated functions, as multiple studies demonstrated that a sample 
size of this size has statistical significance \citep{sample1,sample2}.

On the random selection of the duplicated functions, we decided to put an additional restriction 
on which function pairs are eligible to be selected. The restriction is that we will choose function 
pairs with a similarity metric equal to 100\%. We made this choice because, as shown in \ref{fig:relatory_ex},
there are more than 20,000 lines of code in the AMD Display driver in which the similarity metric 
reaches this value. On all function pairs that respect the restriction, we will choose them 
uniformly at random.

(MAYBE I SHOULD ADD A RESTRICTION ABOUT IGNORING AUTOMATED GENERATED CODE, such as DML folder)

The function pairs selected randomly following our methodology can be consulted in the Table
(CREATE TABLE AND ADD HERE).

CREATE TABLE AND ADD HERE.

\subsection{Ethnographic methodology}

When we make a code duplication mitigation following our systematic approach described in 
\ref{subsec:pipeline}, we are apt to initiate our ethnographic research. We resorted to a
participant-observation approach as our ethnographic methodology, where we will act as 
contributors of the Linux kernel to the AMD Display driver, interacting with the maintainers 
of the AMD Display driver attempting to get our mitigations added to the Linux kernel. 

The Linux kernel communication is made through public mailing lists, where anyone can consult 
the discussions. We plan to collect data artifacts through the AMD Display driver mailing list. 
The mailing list will register our interactions with the Linux community, with their members 
sharing their opinions and suggestions for the mitigation we made, corroborating the discussion 
about the effectiveness of our proposed approach.

Given the time limit of our research, we do not expect to be able to interact with the Linux 
community throughout the successful mitigation we made using the proposed approach, as it passes 
through factors we do not control, which is the time spent waiting for Linux community’s members 
interact with us. For this reason, we resorted to committing to do the participant-observation 
approach in 5 of the successful mitigations and use them for our discussion.
