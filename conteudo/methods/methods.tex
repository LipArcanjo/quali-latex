\en

To answer our proposed research questions, we opted to divide our research. Firstly, we will approach the supplementary research 
RQ1.1, which will serve as a background to our research question RQ1. To answer the RQ1.1, we propose a multiple-method investigation,
by validating our proposed tool in the BigCloneBench databaset \citep{bigclonebench}, which is the standard method used by the
literature, and a empirical analyses of a selected subset of functions in which the tool infers.
With the question RQ1.1 answered, we will return our focus to the RQ1. To answer the RQ1, we opted to create a ethnographic study, 
using participant-observation methodology, we expect to mitigate a selected subset of duplicated functions by refactoring them using
a designed sistematic approach, while iteracting with the Linux community to collect feedback of the mitigation made by us.

\section{Multiple-method Investigation}

The multiple-method investigation to verify the efficiency of our tool in a methodical way is a method in which we consider 
trustworthy.
The first method used is the evaluation of the proposed tool in the BigCloneBench database \citep{bigclonebench}, one of the 
standard code duplication databases used by validation of tool presented in the formal literature. 
As presented by Krinke and Ragkhitwetsagul, the BigCloneBench is not a perfect dataset \citep{bigfail}, and for that reason we opted to do
a additional evalution, in which we will do a empirical analyses on a select subset of functions of the AMD Display driver in which
our tool infers it is a duplication. To a concrete evaluatiion, in both of the methods we will apply a  variation of the similarity 
metric (The similarity metric is a metric measuring the similarity of two code duplications, a proper explanation is made 
in the Chapter \ref{cha:tool}).

\subsection{Evaluation by the formal literature}

The BigCloneBench dataset \citep{bigclonebench} is a collection of Java code with a selection of code duplications pairs categorized by the authors, these
code duplications pairs are not separed solely if they are duplications, the pairs are also categorized by the code duplication
types presented in Subsection \ref{subsec:types}. 

Liut et al presented a sample method to evaluate the their code duplications detection method, in which they sample 20,000 pairs
from each of the code duplication types, with the addition of 20,000 nonduplication pairs to act as the negative samples
\citep{tailor}. To a fairness evaluation, we use the same sample method as them to evaluate our tool.

The proposed tool does not differentiate the code duplication types and infers duplications in the function level instead of the 
code file level, which is the standard in the state-of-art code duplication methods, we were required to adjust the form 
we compute metrics of the evalution. To accomplish this, we will define that every pair of functions in which the similarity metric
is equal or superior to a threshold X will be considered a duplication, and the pair of code files that the functions are contained
are marked as duplication. If on this method, a code duplication pair is corretly indentified, it will count as a correct to its 
code duplication types, if a incorrect duplication pair is infered, it will count as incorrect to all code duplication types. Given
these definitions, we will compute the Recall metric presented in \ref{subsec:codemethods} for each of the code duplication types.

To understand the variation of meaning when we change the similarity threshold, we will eval our tool for the given variations of
the threshold X and discuss the results: 30\%, 40\%, 50\%, 60\%, 70\%, 80\%, 90\%, 100\%.

\subsection{Evaluation by Empirical analyses}

Given the problems of the BigCLoneBench presented by Krinke and Ragkhitwetsagul \citep{bigfail} and the limitations of it in our
scope such as being a Java dataset while the Linux kernel are primarily build in the C programming language, we opted to, together
with the formal litarature evaluation, perform a empirical study analyzing the codes in which our tool infered as duplications. 

The number of code lines and components in the Linux kernel makes it impractical to analyse every code duplication. To limit our
scope, we will analyse solely code duplications in the AMD Display driver. This limitation is still does not reduce the code
duplication to analyse, since in our preliminar tool execution showed in the Figure \ref{fig:relatory_ex}, there is over 20,000
lines of code duplications on the AMD Display driver with similarity metric of 100\%.

For conducting our empiricial analyses, we opted to do a random selection in the code duplications pairs infered by our tool. To
do that, we will use the definition that if the similarity metric of a function pair is greater or equal to X, will be considered a
duplication. To understand the optimal values of X we will analyse 10 functions pair randomic selected 
that have similarity metric X for each of the variations of the parameter X, which are 
30\%, 40\%, 50\%, 60\%, 70\%, 80\%, 90\%, 100\%. 
As it is not expected to have code duplications with similarity exactly a whole number, we will allow a
variation of 1\% for greater or less to each of the variations of the parameter X. 

To do the random selection of functions pairs, we list the code duplications for each of the variations of X and execute 
a random selection algorithm
builtin in in the source code language of our tool. The exactly algorithm of the random selection, along with the selected 
function pairs can be found in Appendix (CREATE APENDICE AND ADD HERE).

When we analyses the selected functions pairs, we will judged empiricaly if they are a duplication or not, then we are not
going further as
categorizing the duplications by code duplication type. We intend to collect the data on this empirical analyses and compare it
to the formal literature method, to undertand if our tool in fact detect duplicated functions and 
the difference it makes when varying the minimum similarity parameter of our tool.

\section{Ethnographic study}

Ethnography is a research method recognized as a significant qualitative empirical approach 
suited to understanding people and cultures, and their associated social and work practices \citep{bookethno}. 
The central tenet of the ethnography approach is to enable research to undertand values, beliefs or ideas shared for a 
 community under study from the members' point of view \citep{ethnosoft}. Thus, the ethnographer is required to act as
a member of  the community, learning and observing deeply what people do and why they do.

Sharp et al discuss the application of the ethnography method in the software engineering research, demonstrating the low adoption 
of the method and its advantages to uncover what practioners do and why they do \citep{ethnosoft}. In our research context, if we
mitigate code duplications by ourselves without iteracting with the Linux community, we cannot undertand if the linux maintainers 
would accept our approach for code duplication and what concerns they would have. Thus, we resorted to a ethnography research, where
we will act as a member of the Linux kernel, begin a contributor with focus on removing code duplications found.

On this section, we will describe a generic approach pipeline to mitigate code duplication found in the Linux kernel. Then we
will describe the methodology to select a subset of duplicated functions found by our tool and finally, describe the ethnographic
methodology to interact with the Linux community.


\subsection{Generic approach pipeline to mitigate code duplication}

The C programming language's way to communicate between source code files is through creation of header files, which is
a library specification \citep{Cbook}. As the Linux kernel is mainly build in C, we resorted to remove code duplications by
creating a library, where the code artifacts, instead of using their duplications, will pass to use the library.

When removing duplicated functions, our desired behavior is to this function to every portion of the code it exist, only allowed a
unique version existent in the library created by us. For this reason, given our initial duplicated function pair, we will use 
the function information functionality(This functionality is described in \ref{subsec:functioncommand}) of our tool to find all 
places that this function is defined in the source code. Thus, we finish with a collection of duplicated function and the collection
of code files in which these functions are defined into.

Given the semantics of a library to containing the shared code by a collection of code files, we added a additional step of the 
approach, which is searching for others functions contained in all of code files of the collection we are working with. 
On this moment, we have a collection of code files, and a non-empty set of functions which are shared by all of the code files 
in the collection.

Finally, for each of the functions shared by all code files, we do a specific-case aproach to mitigate each one of them.
If a function is shared perfectly by all the code files, we use a simply approach to only remove them and create the function in
the source code file, if it is not shared perfectly, we look for refactoring methods which can be applied on this specific case. 

Finishing our mitigation method, if the mitigation ended up working in a initial step, we move it foward to the 
Ethnographic method, iteracting with the Linux community to understand what are their thoughts on the mitigation. If the mitigation
ended up as not working as expected, we will categorized it in two categories, the first one is that the refactoring requires a
decision in which we, as not specialist and non-maintainers of the Linux kernel, cannot make it correctly, although
we believe a specialist would be able to refactoring it. The second categorization is that we do not know if it is in fact possible
to refactoring the duplication. 

The figure (CREATE FIGURE AND ADD HERE) ilustrates the approach pipeline described in this section.





\subsection{Selection of the duplicated functions}

\subsection{Ethnographic methodology}




