%!TeX root=../tese.tex
%("dica" para o editor de texto: este arquivo é parte de um documento maior)
% para saber mais: https://tex.stackexchange.com/q/78101

% As palavras-chave são obrigatórias, em português e em inglês, e devem ser
% definidas antes do resumo/abstract. Acrescente quantas forem necessárias.
\palavraschave{Linux, Duplicação de código, ferramenta}

\keywords{Linux,Code duplication,Tool}

% O resumo é obrigatório, em português e inglês. Estes comandos também
% geram automaticamente a referência para o próprio documento, conforme
% as normas sugeridas da USP.
\resumo{

O kernel Linux é um projeto de software livre amplamente utilizado, alimentando 96,3\% 
dos um milhão maiores servidores web e 85\% dos smartphones em todo o mundo. Dada a sua extensa 
base de usuários, qualquer adição de recursos, correção de bugs ou tratamento de 
vulnerabilidades de segurança pode impactar milhões de pessoas. Manter o kernel é 
uma tarefa enorme, envolvendo mais de 19 milhões de linhas de código e contribuições 
de mais de mil desenvolvedores. Devido à escala do projeto, nem todas as contribuições 
seguem as melhores práticas, resultando em artefatos de código de baixa qualidade que 
podem complicar a manutenção e o desenvolvimento de novos recursos. Um desses artefatos 
é a duplicação de código, que é o foco desta pesquisa. As ferramentas existentes 
para detectar duplicação de código normalmente comparam dois artefatos de código para 
determinar se são duplicatas. No entanto, essas ferramentas não são adequadas para 
identificar duplicações em bases de código em grande escala, como o kernel Linux, nem 
oferecem orientações sobre como resolver as duplicações detectadas. Apesar das 
pesquisas tanto na literatura formal quanto na cinza, as soluções existentes não 
conseguem resolver os desafios específicos do kernel Linux. Para abordar esse tema, 
esta pesquisa foca em dois objetivos principais. 
Primeiro, propõe uma nova ferramenta projetada para identificar duplicações de código dentro 
do kernel Linux, com sua aplicação prática e eficácia avaliadas em subsistemas chave: o AMD 
Display Driver e o subsistema Industrial I/O.
Segundo, analisa as 
duplicações identificadas para extrair padrões e métodos de refatoração, oferecendo 
orientações para futuros contribuidores.

}

\abstract{

The Linux kernel is a widely used Free/Libre/Open Source Software (FLOSS) project, powering
96.3\% of the top one million web servers 
and 85\% of smartphones worldwide. Given its extensive user base, any addition of features, 
bug fixes, or handling of security vulnerabilities can impact millions. Maintaining the kernel 
is an enormous task, involving more than 19 million lines of code and contributions from over 
thirteen thousand developers. Due to the scale of the project, not all contributions adhere to best 
practices, resulting in poor-quality code artifacts that may complicate maintenance and feature 
development. One such artifact is code duplication, which is the focus of this research. 
Existing tools for detecting code duplication typically compare two code artifacts to determine 
if they are duplicates. However, these tools are not well-suited for identifying duplications 
in large-scale codebases like the Linux kernel, nor do they guide on resolving detected 
duplications. Despite searches in both formal and grey literature, existing solutions fall 
short of addressing the specific challenges of the Linux kernel. To address this topic, this 
research focuses on two key goals. 
First, it proposes a new tool designed for identifying code duplications within the Linux kernel, 
with its practical application and effectiveness evaluated in key subsystems: the AMD Display Driver 
and the Industrial I/O subsystem.
Second, it analyzes the identified duplications to extract patterns 
and refactoring methods, offering guidance to future contributors.

}
