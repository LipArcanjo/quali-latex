The Linux Kernel's massive scale, with over 28 million lines of code 
and more than twenty thousand contributors, presents significant maintenance 
challenges; code duplication persists as a notable issue within its codebase, 
potentially hindering its evolution and patching processes. 
Prevailing academic and existing free software approaches to duplication 
detection often lack the scalability for comprehensive codebase analysis or 
offer limited functionality unsuitable for the kernel's complexity, 
generally failing to deliver actionable results or guidance for mitigation 
within its specific context.

In response to these limitations, this research addressed the challenge of 
code duplication within the Linux Kernel through the development of ArKanjo, 
command-line tool for Linux kernel maintenance designed to detect and 
analyze function-level duplications. Released under the MIT license, 
the architectural design of ArKanjo, incorporating a Preprocessor and a 
Query Responder, separates computationally intensive analysis from efficient 
querying for duplications within large codebases. 
Validation procedures, employing both the BigCloneBench dataset and empirical 
analysis conducted on the AMD Display driver, indicated ArKanjo's proficient 
performance in identifying Type-1 and Type-2 duplications, notably when 
similarity thresholds are configured between 80\% and 100\%.

The ethnographic studies, which involved participant observation and a project 
with university students, provided a realistic view of both the opportunities 
and the considerable challenges in mitigating code duplications found by ArKanjo. 
The tool was effectively used by first-time contributors to the Linux kernel to 
identify duplications and prepare patches. While this research included a
successfully merged patch by the author, and successful submissions from many 
student groups, a notable portion of student efforts 
encountered significant hurdles. These difficulties often stemmed from maintainer 
feedback where proposed changes, though reducing duplication, were rejected or 
required substantial rework due to concerns about code readability, added abstraction, 
or the specific context of the code. Technical complexities, such as C macros in 
configuration files, and lengthy patch review times also contributed to the challenges. 
Furthermore, the studies confirmed that maintainers do not always view code duplication 
as inherently negative, especially when it may improve clarity or facilitate 
independent hardware support.

In its entirety, this research contributes the ArKanjo tool as an operational 
method for detecting code duplication, and importantly, provides insights into 
the process of managing such duplications within the Linux kernel subsystems 
studied. The results show that while ArKanjo can effectively identify 
duplicated functions, the path from detection to an accepted mitigation within 
the kernel involves careful consideration of the specific code, established 
community practices, and the maintainers' perspectives on code quality and readability.

\section{Future Work}

The work presented in this thesis suggests several directions for future investigation. 
Enhancements to the ArKanjo tool could be pursued, such as investigating a multi-stage 
detection strategy. This strategy might employ a computationally lighter method for an 
initial, broad pass to identify candidate duplications, followed by more 
resource-intensive, state-of-the-art techniques for refined analysis and improved 
detection of complex clone types. Such an approach could complement efforts for 
improved resource management during its preprocessing phase for very large codebases.
The tool's language support could also be broadened by
developing parsers for additional programming languages, building upon its adaptable
architecture. Beyond the tool itself, future research could extend the empirical
analysis of code duplication and mitigation strategies to a broader range of Linux kernel 
subsystems or other large-scale C projects, to determine the wider 
applicability of the observations made.
