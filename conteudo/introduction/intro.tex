\en

\section**{Problem outline}
\label{sec:problem}

Representing 3.77\% of the personal desktops and 70,69\% of the smartphones worldwide, the Linux kernel 
is a open source project providing value to billions of people in their daily lives. 
The magnitude that the project has reached over the years makes the addition of features, as well as handling bugs and security 
vunerabilities, potentially affect a significant number of people, demonstrating the importance of maintaining the 
kernel. 

The maintainment of the Linux kernel is a complex task, as the project contains more then 19 million lines of code and have more
then thirteen thousand contributors along the lifespan of the project \citep{linuxquantity}. Given the magnitude of the project,
it is unrealistic to expect that all contributors followed the best programming practices and principles throughout every 
contributions they made to the project. It is expected to find a good amount of bad quality code artifacts in the source code, 
which could result in hindering or compromising the maintenance of this project, as well as the addition of new features for it.

Device driver, as stated by Madieu, are a "piece of software whose aim is to control and manage a particular hardware device, hence 
the name device driver" \citep{driverdef}. Device drivers are major part of the Linux kernel, representing 66 \% of the source code and
responsible for the management of crucial element such as keyboards, mouses and GPU \citep{marcelo}. The moment the kernel addopts
to support new hardwares, it requires to make alteration of existent device drivers or the creation of new ones, 
this makes the maintainment of the devices drivers a important duty of the Linux community.

The AMD Display driver of the Linux kernel is driver responsible to enable AMD GPU's to operate correctly in desktop linux
environments, which is a important duty, given that AMD GPU represented 19 \% of the personal GPU market in 2023 \citep{gpumarket}.
When we iteracted with the maintainers of the AMD Display driver, they shared with us their pain with one of the characteristics
bad quality code artifacts, that is duplicated code artifacts, which hinders the maintainment of the driver. 


Searching for solutions to the AMD Display driver problem with duplicated code artifacts, we were able to find tools that given 
two code artifacts, infer if they are duplication of each other, and the formal literature in the code duplication mainly focused 
in optimizing the accuracy of these tools in this given task. These solutions do not approach our problem, as we are not able to give them a codebase
and they returns to us the code duplication in the codebase, and they do not guide us to how mitigate the code duplications after
we detected them, requiring us to search for others alternatives.

Given the unsuccessful results on our serach in the formal literature, we started to search for solutions in the gray literature, 
but we ended up without success. In the gray literature, the best solutions we found were primitive solutions that finds every pair
of code files that are duplications. As the AMD Display driver is a driver created in 20XX (TODO HERE) and maintained until today,
the code 
files that contains duplicated code normally also have specific code that is only implemented in this file. Thus, 
a solution that finds duplication in a more deeply context is necessary to correct find duplicatons in the Linux kernel context.

\section**{Research Design}

The code duplication problem outlined in Section \ref{sec:problem} cannot be resolved by the solutions presented in the formal 
literature nor the gray literature, we intend to propose an approach to indentify and mitigate the code duplications in the 
Linux Kernel. For that, we propose a tool capable to indentify code duplications in the kernel, and we mitigate code duplications in the kernel. 
On this research, we expect to extract patterns of software engineers techniques, such as refactoring methods, which we commonly 
used to mitigate the code duplications, thus, serving as guidance to future contributors to approach the code duplications problem 
in the given context. To properly investigate our proposed approach, we developed the research questions below to guie our research.

\textbf{RQ1:} How effective is our approach in mitigating duplicate code in the Linux kernel?

\textbf{RQ1:} How effective is our approach in identifying duplicate code in the Linux kernel?

This research is divided into two phases. On the \textbf{Phase I}, we will propose, build and validate a tool capable to indentify code 
duplications in function level for the C programming language, we expects this proposed tool to indentify code duplications
in the Linux kernel, as in our preliminary investigation the code files in the kernel is commonly a union of duplicated and 
nonduplicated functions. To validate the tool, we will do a triangulation of results by evaluate our tool in the code duplication
databases used in the formal literature and (DO NOT KNOW IF I REALLY WILL DO THIS) doing a empirical analyses in a set
of randomly selected codes of the AMD Display driver which our proposed tool claims that is a duplications. 
The detailed explication of how we will validate our tool is demonstrated in the Chapter \ref{cha:method}.

Given the first phase completion, we would have a validated tool to indentify code duplications in the Linux kernel 
context and we can move on to how mitigate these duplications, which comprehend the second phase of our research. On the 
\textbf{Phase 2}, we intend to discover ways to mitigate the code duplication detected in the linux kernel. 
To achieve our objective, we designed a etchnographic approach, which enables us to interact with the linux community
to collect artifacts validating or refuting our mitigation. The figure (CREATE A FIGURE HERE) ilustrate the research
process we addopted in this research.

(POINTS TO CONSIDER: I do not like much this introduction, but is the best I could do)
(I DO NOT HAVE MORE A PLACE TO FIT THIS PARAGRAPH) An additional usefulness of our proposed approach, which we do not verify
in our research, is to give a new method to people become 
contributors to the Linux kernel. The process of starting to contribute to the Linux kernel can be unintuitive, which can cause 
people to distance themselves. Our approach provides to the community a new form of newcomers to start contributing by removing 
simple code duplications, while they learn the whole process of being a Linux contributor.

\section**{Manuscript Structure}

This manuscript consists of four more chapters. 
Chapter \ref{cha:back} presents the literature overview for the code duplication detection (main definitions, current approches 
methods in the literature), a briefly description of the Linux kernel and a review of refactoring methods used throughtout 
this research.
Chapter \ref{cha:tool} presents our proposed tool to detect code duplication in the Linux kernel, describing all of the main 
components.
Chapter \ref{cha:method} describes the research methods selected to guide our work.
Chapter \ref{cha:results} presents the current research stage and the work plan, with addition of preliminary results obtained.



