\en

This chapter presents an overview of fundamental topics needed to understand 
the research, such as the Linux Kernel, refactoring, types of code duplication, 
code duplication detection methods presented in the literature, and ways to 
evaluate these methods. This foundation will make it easier to grasp the 
concepts discussed in the research.

\section{The Linux Kernel}
Linus Torvalds released the initial version of the Linux kernel, a Unix-like operating system kernel, in 1991. It was initially released under a license of Torvalds' creation, but the Linux kernel was later relicensed under GNU General Public License version 2 (GPLv2) in 1992.
%
It began gaining popularity in the late
1990s~\citep{linuxbook}. Contrary to popular belief, Linux is not a
complete operating system, as it lacks components like filesystem utilities and
windowing systems. The so-called ``Linux distributions'' commonly
recognized by the public are GNU/Linux operating systems, which
include various essential applications alongside the Linux kernel,
such as system~\citep{gnuref}. Examples of Linux distributions are Ubuntu 
and Arch Linux.

The Linux kernel is monolithic and organized into subsystems, such as
the process scheduler, memory management, device driver infrastructure,
networking, and filesystems~\citep{melissa}. Each subsystem typically has a
dedicated maintainer or a team of maintainers responsible for overseeing its
development and managing contributions~\citep{melissa}.

Development contributions to the Linux kernel are coordinated using the Git
version control system. These contributions are formatted as patches—text
documents that outline the differences between two source code versions.
These patches are then submitted and reviewed via the Linux mailing lists~\citep{melissa}.

Mailing lists are the primary means of communication within the Linux
community. Community members use them to submit and review patches,
discuss Linux-related topics, and debate potential new features. Given the
kernel extensive modularity, each subsystem has its mailing list, allowing
discussions to be organized around specific areas of development~\citep{melissa}. All interactions on these mailing lists are documented and
archived in the Linux Kernel Lore\footnote{\url{https://lore.kernel.org}}, which serves as a
repository for kernel discussions and can be accessed by anyone interested.

One of the subsystems within the Linux kernel is the AMD Display driver, which
is responsible for implementing the drivers needed for AMD GPUs to function
correctly in a Linux environment.
%
During the exploratory phase of this work, in a brainstorming conversation with the maintainers of this subsystem, they reported a practical issue: a significant amount of duplicated code, which complicates the maintenance of this subsystem.
%
This problem serves as the
starting point for our research, with the AMD Display driver as our primary
object of study as we aim to understand and mitigate code duplication in the
Linux kernel. The AMD Display driver subsystem contains more than 391,000 lines 
of code and more than 1,089 code files
\footnote{ 
Data extracted directly from the Linux repository with the help of the cloc tool~\citep{cloc}.
}, 
making it a considerable component within the Linux kernel. Like other 
subsystems, discussions about the AMD Display driver are conducted on its 
designated mailing list, which can be accessed here: 
\url{https://lists.freedesktop.org/mailman/listinfo/amd-gfx}.

Another subsystem within the Linux kernel explored in this work is the
Industrial I/O subsystem (IIO). According to the Linux kernel documentation, 
``The main purpose of the Industrial I/O subsystem (IIO) is to provide support 
for devices that in some sense perform either analog-to-digital conversion (ADC) 
or digital-to-analog conversion (DAC) or both''~\citep{iiodoc}. The IIO subsystem 
contains more than 281,772 lines of code and more than 755 code files
\footnote{
Data extracted directly from the Linux repository with the help of the clol tool~\citep{cloc}.
}.
We opted to include the IIO subsystem in our experiments to ensure our findings 
are not limited to one subsystem. Additionally, we have an established point of
contact within the IIO subsystem to facilitate this work.
Like other subsystems, discussions about the IIO subsystem are conducted on its 
designated mailing list, which can be found here: 
\url{https://subspace.kernel.org/vger.kernel.org.html}.

