\en

\section{The Linux Kernel}

The Linux kernel is an Unix-like operating system published by Linus Torvalds in 1992 as a free software and
it started becoming popular in late 1990s \citep{linuxbook}. Against popular belief, linux is not a full 
operating system as it do not includes applications such as filesystem utilities, windowing systems. The "Linux" 
distributions know to the public are in fact GNU/Linux operating systems \citep{gnuref}.

The Linux kernel is monolithic organized as a set of subsystem such as process scheduler, memory management, 
device driver infrastruture, networking and filesystems \citep{melissa}.
Each subsystem usually have a maintainer or a group of maintainers, which are people responsible 
for managing and accepting contributions into the subsystem they care of \citep{melissa}. 

The contributions to the Linux are coordinate with the usage of the Git source management tool. The contributions 
made to the project are formated as patches, which are text documents describing differences between two different 
versions of a source tree  The patches are sent and reviewed through the Linux mailing lists \citep{melissa}.

Mailing lists are the official means of communication of the Linux community, through the mailing lists the community members 
send and review
contributions to the kernel, discuss topics related to the Linux such as which new features to add, et cetera. 
As there are multiples
subsystems in the Linux kernel, there are multiples mailing lists in the Linux to organize the discussion made. Every subsystem 
have their own mailing list to concentrate the discussion related to the subsystem \citep{melissa}. The iteractions made in the
mailing lists are documented and stored in the Linux kernel Lore \citep{linuxlore}, which serves as a repository of the
discussion made in the Linux kernel to be consulted by anyone interested.

One the the Linux kernel subsystem is the AMD Display Driver, a subsystem responsible for implement the drivers required to enable
AMD GPU's work properly in the Linux environment. The maintainers of this subsystem reported a empirical problem they have in
the subsystem, which is a significant amount of duplicated code in the subsystem, a problem that hinders the maintainment of the
subsystem. This reported problem is the starting point that initiated our research, making the AMD Display Driver our principal
objecte of study to undertand how to mitigate duplicated code in the Linux kernel.


(I WANT TO CHANGE THE AMD DISPLAY paragraph, WHY DO I NEED TO CHANGE IT? I SHOULD ANOTATE WHAT I WANTED, I MADE A CHANGE
 TO MAKE IT BETTER BUT NOT SURE IF THERE IS SOME INFORMATION I WANTED TO PUT IN IT)

