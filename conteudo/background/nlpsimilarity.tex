\en

\section{Text Similarity Detection Methods}
\label{sec:similarity}

The standard methodology for text similarity detection is based on building
vector embeddings of texts and applying a distance function to these embeddings
to measure their similarity. The literature proposes multiple vector
embedding methods, such as bag-of-words, TF-IDF, LSI, and Word2Vec
\citep{gensimlivro}. Recent works related to vector embedding construction
utilize Large Language Models (LLMs), as exemplified by Ethayarajh's work,
which employs BERT, ELMo, and GPT \citep{llmsimilar}. The most popular distance
function used in the literature is cosine similarity, defined by the formula
\citep{cosineref}:

$$\text{cosine similarity} = SC(A,B) = \frac{ A \cdot B}{ \lVert A \rVert \lVert B \rVert }$$

Where $A$ and $B$ are the vector embeddings of the two texts. An advantage of
cosine similarity is that $SC(A,B)$ produces a number between $0$ and $1$,
where values closer to $1$ indicate higher similarity. It is common to
interpret cosine similarity as an approximation of probability, transforming
values between $0$ and $1$ into percentages between $0\%$ and $100\%$.
Throughout our research, we will use this percentage form.

It is well-known that LLMs have a high computational cost. For instance, an
analysis by Reimers and Gurevych found that finding the most similar pair in a
collection of 10,000 sentences required about 65 hours with BERT
\citep{bertsimilar}. Since we aim to compare pairs of similar magnitude in
our work, we decided not to use LLMs for our vector embedding method, even
though they may be suitable for smaller codebases.

\subsection{Gensim}

Gensim is an FLOSS Python library \citep{gensim}, which its creators
describe as "The fastest library for training of vector embeddings - Python or
otherwise. The core algorithms in Gensim use battle-hardened, highly optimized,
and parallelized C routines." \citep{gensimsite}. Due to Gensim's speed, we
chose this library for our vector embedding method.

In this work, we do not explore further alternatives for vector embeddings, as
optimizing computational cost or achieving state-of-the-art code clone
detection is not the primary objective of our research.
