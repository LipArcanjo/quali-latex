\en

\section{Code Duplication Detection}

Code Duplication Detection is a field in computer science studies that gets the attention of researchers back to 1988 \citep{firstman}.
The occurrence of Code Duplication is a harmful artifact to have in a software, affecting software related tasks such as code readability,
introdution of bugs, etc.   \citep{harmone}. 
On most of the cases, the occurrence of code duplication tends to create a more unstable software then the nonduplicate code counterpart.   \citep{harmtwo}

Through the years, the subject of studies about Code Duplication Detection branched out mainly in two fields of study,
Code Clone Detection and Code Plagiarism Detection,
with the former one focused on the technical aspect while the later one also introduces an social aspect on the field of research
\citep{litreview}. For our research we are only interested on the Code Clone Detection branch given the nature of our work.

\subsection{Types of code clone duplication}

The categorization of two code artifacts as a duplication of each other is a subjective task (REFERENCE HERE). 
To mitigate this human aspect on the given problem, the literature classified code duplication on the scale of changes between 
the code artifacts in four main types of code clone duplication \citep{litreview}. Given two code artifacts F and G, 
the types of Code Duplication are described as follow \citep{litreview}:

\begin{itemize}
	\item \textbf{Type-1:} The differences between F and G are in the context of revision of comments, variable names, white spaces and any other kind of irrelevant elements. (ADD FIGURE) shows a typical example of Type-1 Code Duplication \citep{litreview}. 

	\item \textbf{Type-2:} The differences between F and G is the same as the type-1, with the addition of considering addition and deletion of redudant codes. (ADD FIGURE) shows a typical example of Type-2 Code Duplication \citep{litreview}. 

	\item \textbf{Type-3:} The difference between F and G is the same as the type-2, with the addition of considering the reorder of code blocks as well as statements within code blocks. (ADD FIGURE) shows a typical example of Type-3 Code Duplication \citep{litreview}. 

	\item \textbf{Type-4:} The difference between F and G is the same as the type-3, with the addtionf of considering changes on data sctrucure, the order of operands/operators in expresssions, or replacing part of codes with equivalent composition. (ADD FIGURE) shows a typical example of Type-4 Code Duplication \citep{litreview}. 
\end{itemize}

\subsection{Literature approches for code clone detection}

Given the passage of time, the literature developed many techniques and methods to approach the code clone detection problem. 
For better understanding and analyze of these approachs, the literature divided the researches done in
five main methodologies given the main nature of their work \citep{litreview} , which are:

\begin{itemize}
	\item \textbf{Textual-based approaches:} The literature review made by Chen \citep{litreview}  states that this methodology
	is the first to be investigated by the literature. This line of methodology is based on see code as a text artifact and apply
	textual approachs for text similarity. One example of this methodology applied is the work of Roy and Cordy \citep{textexample}, 
	which uses parsing and grammar techniques to detect code clones.

	\item \textbf{Token-based approaches:} This methodoly tries to approach the problem by seeing the code in a similar way to
	the lexical analysis of the compiling process \citep{litreview} . That means doing work in the line of recognizing constants, keywords and other 
	specific programming languages tokens, mapping variables to <pair,value> representations to become naming-independent, etc. One example of 
	this methodology applied  is the work of Toomey et al. \citep{tokenexample} to detect code clones through hashed token sequences.

	\item \textbf{Tree-based approaches:} While the token-based approach uses the lexical analysis of the compiling process, the 
	tree-based approach uses the syntax tree, the structure that  stores the semantic information of a code artifact in respect of the 
	programming language \citep{compiler}. One example of research that uses this information to detect code clones is the one done by
	Chilowicz and Duris \citep{treeexample}.

	\item \textbf{Metric-based approaches:} This methodology approches the problem by extracting metrics from the code artifact and use 
	those metrics as a embedding representation of the artifact to measure similarity \citep{litreview}, similarly to the famous  
	Word2Vec work done by Mikolov et al \citep{wordtovec}. Examples of metrics can be program size, number of variables, number of 
	access to memory, etc. One example of this methodology applied is the work done by Kaur and Sharma. \citep{metricexample}

	\item \textbf{Graph-based approaches:} Similarly to the Tree-based approaches, this methotodology depends heavily on an intermediate
	program representation, called program dependence graph (PDG) \citep{prodg}. The representation store information 
	about program's control, statement and data dependencies. One example of usage of the program dependence graph is 
	the recently work done by Liu et al. \citep{tailor}.
\end{itemize}

Given the necessity to evaluate and compare approches for code clone detection, it was developed and widely-adopted two datasets: 
OJ-Clone, a dataset collected from a pedagogical online judge system \citep{ojclone}, and BigCloneBench, a datatset collected from over 
25,000 Java software systems \citep{bigclonebench}. The existence of only two spreaded datasets is a open concern in literature as 
the limited scope of OJ-Clone and the problems of BigCloneBench presented by Krinke and Ragkhitwetsagul \citep{bigfail}

The state-of-art methods for code clone detection usually have high computational time and memory space usage, as well being
programming language specific approches, which cannot be easily implemented in programming languagues that were not used in 
their research \citep{litreview}. These are main concern for our research and for this reason we will use a generalist textual-based 
approach based solely on Natural Languague Processing (NLP) text similarity detection methods, which we will introduce and compare
with the literature approches later on this document. 












