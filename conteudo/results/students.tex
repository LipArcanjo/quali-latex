\section{Students}

As presented in chapter \ref{cha:method}, we propose to students to mitigate duplications
found by the ArKanjo tool and send patches to the linux subsystems maintainers, while
documenting their experiences and learning through the process. 

For the students that worked
in the IIO subsytems, we gave them duplicated function pairs found by the tool and their 
responsability was to think of how to mitigate the duplication and sent the patch to the 
maintainers. The student that worked in the AMD Display driver have the additional work
to use the tool to find the duplications.

Appendix \ref{app:stuart} has references
to the artifacts discussed in this section, including the duplication function pairs that students 
mitigate, the students's reports, the patch students sent to the linux kernel, and the 
answers of the questionary asked to the student that worked in the AMD Display driver and the 
teaching assistant that used the tool to find duplications in the IIO.

\begin{table}
\begin{tabular}{ | c | c | c | c | m{6em} | }

\hline

\textbf{Group} & \textbf{Subsystem} & \textbf{Lines added} & \textbf{Lines removed} & \textbf{Refactoring methods used}
\\ \hline 

1 & IIO & 7 & 14 & None \\ \hline
2 & IIO & 23 & 24 & Parameterize method \\ \hline
3 & IIO & 12 & 30 & Parameterize method \\ \hline
4 & IIO & 2 & 7 & None \\ \hline
5 & IIO & 17 & 70 & Parameterize method \\ \hline
6 & IIO & - & - & Parameterize method \\ \hline
7 & IIO & 21 & 22  & None \\ \hline
8 & IIO & - & - & None \\ \hline
9 & IIO & 33 & 52 & Parameterize method \\ \hline
10 & IIO & 32 & 58 & Parameterize method \\ \hline
11 & IIO & - & - & Parameterize method \\ \hline
12 & AMD & 89 & 489 & None \\ \hline

\hline
\end{tabular}
\caption{Refactor methods used and metrics of the students's mitigations.}
\label{tab:stu}
\end{table}


Table \ref{tab:stu} summarizes the mitigation approachs of students's groups in terms 
of lines changed and refactor methods used. A total of seven groups applied the parameterize refactor method,
and two groups applied the extractor method on their approaches to mitigate the duplications. 
The other three groups does not applied any refactor methods presented in the literature. 
We discuss the groups's approaches and the subsystem maintainers opinions on the patches that students
sent.

Groups one and four have worked in similar duplications, where there are two functions returning a boolean mirroring
the results. That means, when the first function returns true, the second function returns false, and when the first
functions returns false, the second function return false. The group one have a simple approach and the maintainers 
approved the changes quickly. The group four attempted to mitigate the duplication with macros, which is a method 
found in the code file they were working. The maintainers disagree with the proposed approach, arguing 
that the proposed changes decreases the code readability and asked for a approach similar to the group one. 
The group four submitted the patch after coding the approach proposed by the maintainers.

Group three, six and ten created a struct to encapsulate the parameters to mitigate the duplication when 
applying the parameterize method. The group three mitigation was send to the maintainers and the approved 
without significant comments, while the patch from group ten was reject by the same maintainer that approved
the patch from group three. The maintainer justification to reject the patch is that group ten refactored 
code files that has simple context, threfore adding a layer of abstraction to remove a duplication 
deacrease the code readability without clear gains on the codebase quality.

Group seven and eight used the extract method to approach the mitigations. The group eight send the 
mitigations to the maintainers and the code changes was approved with minor requested changes
about code style and smaller errors fix. For the group seven mitigation, the maintainers reject 
their patch because the code changes adds layers of abstractions on the code, turning it more complex
and harder to read. The maintainers proposed to the group seven to remove the duplicated functions 
and insert the duplicated code multiple times were the functions were being called. 
This suggestion we could think that it is a bad practice for code quality, as adds more complexity and 
responsibality to a bigger function, it is view as the best choice for removing layers of abstraction 
and making the code easier to read.

Group nine and eleven faced similar issue as us, where configurations files
designed as macros complicates to approach the mitigations. Group nine applied the parameterize method
to not need to know the configurations in the generic function, which is reasonable as the duplication
approach only depends on one configuration value. Group eleven worked with multiple configuration values,
and to refactor the duplication they proposed to create a bigger macro that encapsulates the other macros,
which is a clever solution but adds abstraction and complexity on macros, which makes the code harder
to debug and read. We do not get the feedback of the maintainers on the group eleven approach to know their 
opinions.

Group twelve is the student that executed the tool and pass throught the process of findings duplications.
The student found a duplication in the driver that happens accross eighteen code files, creating a patch
of removing four hundreds lines of duplicated code. 
The student send the mitigations to the maintainers and their feedback pointed to fix warning in the code, 
without arguing about removing the duplications.

Analysing the students approaches for the duplications and the maintainers feedback, 
we saw that people without previous experience in the linux kernel development could approach 
duplications found by the ArKanjo tool and become contributors to the kernel. We observed that 
not all duplications are view as code with bad quality, with the maintainers analysing the trade off
between what is the purpose of the duplicated code, the code readability, and the levels of abstraction 
added to address the duplications.

\subsection{Questionary Answers}

BLABLABLA
