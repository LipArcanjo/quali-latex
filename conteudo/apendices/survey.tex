\chapter{Survey Results}

\label{app:survey}

This appendix includes the questions asked in the survey for the teaching assistant and the
student that used the ArKanjo tool. Teaching assistant's answers are showed as answer one and 
the student's answers are showed as answer two.

\paragraph{Question 1}

\

\textbf{Question:} How was your experience with the installation process of the tool? 
If there were any difficulties during the installation, please indicate them. 
Please rate the tool from one to five in regard to this aspect.

\textbf{Answer 1:} At first I've tried to install it on Arch Linux
(which is not officially supported by ArKanjo), and had some trouble to find and install 
the corresponding dependencies. Then, I've decided to run it on an Ubuntu container and 
had no problems with the installation. Score: 5 out of 5.

\textbf{Answer 2:}

\paragraph{Question 2}

\

\textbf{Question:}How was your experience with the documentation and using the tool in general? 
Do you feel there were any missing or unclear parts in the documentation? If so, which ones?
Rate the tool  from one to five related to the documentation.

\textbf{Answer 1:} There were some misleading information about the usage of some flags, 
but Luan clarified it all and committed to correct the documentation. These problems, 
regarding similarity percentage for queries, didn't cause significant issues to use ArKanjo, 
but limited my workflow a bit. Score: 4 out of 5.

\textbf{Answer 2:}


\paragraph{Question 3}

\

\textbf{Question:}Regarding the preprocessing step to find duplicates, where you pass the codebase. 
Did you experience any difficulties with this process? How was the user experience? 
Was it clear enough what the process was doing and why there might be a delay in execution?
Rate the tool  from one to five related to the preprocessing step.

\textbf{Answer 1:} Running the preprocessor was intuitive and the documentation covered 
it all correctly. My only complaint is that I was only able to preprocess with the gensim method. 
The diff strategy kept running for hours and didn't complete when I gave up. Score: 4 out of 5.

\textbf{Answer 2:} 

\paragraph{Question 4}

\

\textbf{Question:} Regarding executing the commands implemented in the tool, did you 
have difficulty understanding what the tool's output represented and how to execute the commands?
What is your opinion on the formatting and presentation of the output? Would you change anything 
in the output presentation that you believe would improve the experience?
Rate the tool  from one to five related to executing and understanding the output of the commands.

\textbf{Answer 1:} The command outputs were very intuitive and easy to analyze. My only complaint 
is that some of it is colored and this is problematic to output directly onto a file for further analysis, 
since the color chars are preserved. I would suggest a new flag that decides if output is highlighted 
with colors, and either default it to no-color or having a clear documentation about the output colors.
Score: 4 out of 5.

\textbf{Anyyswer 2:}

\paragraph{Question 5}

\

\textbf{Question:} Did the tool meet your expectations? Do you believe it effectively 
fulfills its purpose, that is, detecting code duplications?

\textbf{Answer 1:} It did fill my expectations. I've collected a bunch of code duplications with ArKanjo,
presented the most promissing results to other people and most of them resulted in merged patches to the 
IIO and DRM subsystems.

\textbf{Anyyswer 2:}

\paragraph{Question 6}

\

\textbf{Question:} Based on your current knowledge of the tool, do you have any ideas 
for new features that you consider relevant or interesting?

\textbf{Answer 1:} I would suggest polishing the diff heuristic to find duplications and/or add new ones. 
Setting output colors with flags would be a good addition. Flags to highlight duplications that happen on 
a same file or directory would also be good.

\textbf{Anyyswer 2:}

\paragraph{Question 7}

\

\textbf{Question:} Please provide any additional comments or insights you would like to 
share that have not been covered in the previous questions.

\textbf{Answer 1:} For some decentralized codebases, where a directory is maintained by different people 
and open source communities, such as the Linux Kernel, it would be interesting to be able to filter 
duplications that happen on a same file or directory. When using ArKanjo, I had to use external scripting 
and AI to do this, and it would be great if ArKanjo had built-in features for output analysis.
Otherwise, great tool that helped the improvement of code quality and reduce technical debt.

\textbf{Answer 2:}
