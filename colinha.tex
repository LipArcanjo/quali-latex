% Arquivo LaTeX de exemplo de artigo
%
% Criação: Jesús P. Mena-Chalco
% Revisão: Fabio Kon e Paulo Feofiloff
% Adaptação para UTF8, biblatex e outras melhorias: Nelson Lago
%
% Except where otherwise indicated, these files are distributed under
% the MIT Licence. The example text, which includes the tutorial and
% examples as well as the explanatory comments in the source, are
% available under the Creative Commons Attribution International
% Licence, v4.0 (CC-BY 4.0) - https://creativecommons.org/licenses/by/4.0/


%%%%%%%%%%%%%%%%%%%%%%%%%%%%%%%%%%%%%%%%%%%%%%%%%%%%%%%%%%%%%%%%%%%%%%%%%%%%%%%%
%%%%%%%%%%%%%%%%%%%%%%%%%%%%%%% PREÂMBULO LaTeX %%%%%%%%%%%%%%%%%%%%%%%%%%%%%%%%
%%%%%%%%%%%%%%%%%%%%%%%%%%%%%%%%%%%%%%%%%%%%%%%%%%%%%%%%%%%%%%%%%%%%%%%%%%%%%%%%

% A opção twoside (frente-e-verso) significa que a aparência das páginas pares
% e ímpares pode ser diferente. Por exemplo, as margens podem ser diferentes ou
% os números de página podem aparecer à direita ou à esquerda alternadamente.
% Mas nada impede que você crie um documento "só frente" e, ao imprimir, faça
% a impressão frente-e-verso.
%
% Aqui também definimos a língua padrão do documento
% (a última da lista) e línguas adicionais.
%\documentclass[12pt,twoside,brazilian,english]{article}
\documentclass[10pt,twoside,english,brazilian]{article}

% Ao invés de definir o tamanho das margens, vamos definir os tamanhos do
% texto, do cabeçalho e do rodapé, e deixamos a package geometry calcular
% o tamanho das margens em função do tamanho do papel. Assim, obtemos o
% mesmo resultado impresso, mas com margens diferentes, se o tamanho do
% papel for diferente.
\usepackage[a4paper,landscape]{geometry}

\geometry{
  textheight=192mm,
  textwidth=277mm,
  vmarginratio=1:1,
  hmarginratio=1:1,
  headsep=11mm, % distância entre a base do cabeçalho e o texto
  headheight=21mm, % qualquer medida grande o suficiente, p.ex., top - headsep
  footskip=10mm,
  marginpar=20mm,
  marginparsep=5mm,
}

% Vários pacotes e opções de configuração genéricos; para personalizar o
% resultado, modifique estes arquivos.
\usepackage{imegoodies}

%%%%%%%%%%%%%%%%%%%%%%%%%%%%%%%%%%%%%%%%%%%%%%%%%%%%%%%%%%%%%%%%%%%%%%%%%%%%%%%%
%%%%%%%%%%%%%%%%%%%%%%%%%% ESPAÇAMENTO E ALINHAMENTO %%%%%%%%%%%%%%%%%%%%%%%%%%%
%%%%%%%%%%%%%%%%%%%%%%%%%%%%%%%%%%%%%%%%%%%%%%%%%%%%%%%%%%%%%%%%%%%%%%%%%%%%%%%%

% LaTeX por default segue o estilo americano e não faz a indentação da
% primeira linha do primeiro parágrafo de uma seção; este pacote ativa essa
% indentação, como é o estilo brasileiro
\usepackage{indentfirst}

% A primeira linha de cada parágrafo costuma ter um pequeno recuo para
% tornar mais fácil visualizar onde cada parágrafo começa. Além disso, é
% possível colocar um espaço em branco entre um parágrafo e outro. Esta
% package coloca o espaço em branco e desabilita o recuo; como queremos
% o espaço *e* o recuo, é preciso guardar o valor padrão do recuo e
% redefini-lo depois de carregar a package.
% TODO: depois que ubuntu 18.04 se tornar obsoleta (abril/2023), remover
%       as linhas "oldparindent" e carregar a package com a opção "indent".
\newlength\oldparindent
\setlength\oldparindent\parindent
\usepackage[parfill]{parskip}
\setlength\parindent\oldparindent

\usepackage[scale=.85]{sourcecodepro}

\usepackage[mono=false]{libertinus}
% Com LuaLaTeX/XeLaTeX, Libertinus configura também
% a fonte matemática; aqui só precisamos corrigir \mathit
\ifLuaTeX
  \setmathfontface\mathit{Libertinus Serif Italic}
\fi
\makeatletter
\ifXeTeX
  % O nome de arquivo da fonte mudou na versão 2019-04-04
  \@ifpackagelater{libertinus-otf}{2019/04/03}
      {\setmathfontface\mathit{LibertinusSerif-Italic.otf}}
      {\setmathfontface\mathit{libertinusserif-italic.otf}}
\fi
\makeatother

\ifPDFTeX
  % A família libertine por padrão não define uma fonte matemática
  % específica para pdfLaTeX; uma opção que funciona bem com ela:
  %\usepackage[libertine]{newtxmath}
  % Outra, provavelmente melhor:
  \usepackage{libertinust1math}
\fi

%\nocolorlinks % para impressão em P&B
\input{extras/source-code}

% Diretórios onde estão as figuras; com isso, não é preciso colocar o caminho
% completo em \includegraphics (e nem a extensão).
\graphicspath{{figuras/},{logos/}}

% Comandos rápidos para mudar de língua:
% \en -> muda para o inglês
% \br -> muda para o português
% \texten{blah} -> o texto "blah" é em inglês
% \textbr{blah} -> o texto "blah" é em português
\babeltags{br = brazilian, en = english}

% Bibliografia
\usepackage[
  style=extras/plainnat-ime, % variante de autor-data, similar a plainnat
  %style=alphabetic, % similar a alpha
  %style=numeric, % comum em artigos
  %style=authoryear-comp, % autor-data "padrão" do biblatex
  %style=apa, % variante de autor-data, muito usado
  %style=abnt,
]{biblatex}

%%%%%%%%%%% Personalização para este arquivo especificamente %%%%%%%%%%%

% Espaçamento simples
\singlespacing

\setlist[description]{itemsep=0pt,parsep=1pt,leftmargin=0pt}

\setlength{\parskip}{0pt}
\setlength{\parindent}{0pt}

\usepackage{titlesec}

\titleformat{\section}[hang]
  {\scshape\bfseries}
  {\thesection}
  {0.7em}
  {}

\titlespacing{\section}
  {0pt}
  {1.5\baselineskip plus 1.5\baselineskip minus 1\baselineskip}
  {.5\baselineskip plus .5\baselineskip minus .2\baselineskip}


%%%%%%%%%%%%%%%%%%%%%%%%%%%%%%%%%%%%%%%%%%%%%%%%%%%%%%%%%%%%%%%%%%%%%%%%%%%%%%%%
%%%%%%%%%%%%%%%%%%%%%%%%%%%%%% INÍCIO DO CONTEÚDO %%%%%%%%%%%%%%%%%%%%%%%%%%%%%%
%%%%%%%%%%%%%%%%%%%%%%%%%%%%%%%%%%%%%%%%%%%%%%%%%%%%%%%%%%%%%%%%%%%%%%%%%%%%%%%%

% O arquivo com os dados bibliográficos para biblatex; você pode usar
% este comando mais de uma vez para acrescentar múltiplos arquivos
\addbibresource{bibliografia.bib}

% Este comando permite acrescentar itens à lista de referências sem incluir
% uma referência de fato no texto (pode ser usado em qualquer lugar do texto)
%\nocite{bronevetsky02,schmidt03:MSc, FSF:GNU-GPL, CORBA:spec, MenaChalco08}
% Com este comando, todos os itens do arquivo .bib são incluídos na lista
% de referências
%\nocite{*}

\pagestyle{empty}

\begin{document}

{
  \centering
  \Large\bfseries

  \rule[.6ex]{.2\textwidth}{1pt}
  \quad\space Colinha essencial de \LaTeX \quad\space
  \rule[.6ex]{.2\textwidth}{1pt}\par
}

\setlength{\columnsep}{20pt}
\setlength{\columnseprule}{.2pt}
\begin{multicols}{3}

\section*{Capa/título}

\textbackslash{}title\{O título\}

\textbackslash{}author\{O autor\}

\textbackslash{}maketitle\quad (gera a página de título)


\section*{Divisões}

\textbackslash{}part\{nome\}

\textbackslash{}chapter\{nome\}

\textbackslash{}section\{nome\}

\textbackslash{}subsection\{nome\}

\textbackslash{}subsubsection\{nome\}

\textbackslash{}paragraph\{nome\}

\textbackslash{}subparagraph\{nome\}

\textbackslash{}section*\{\},
\textbackslash{}chapter*\{\} etc.\quad
(elimina numeração)


\vspace{\baselineskip}


\textbackslash{}footnote\{texto da nota\}


\section*{Modo matemático}

(veja também \textsf{texdoc undergradmath})


\vspace{\baselineskip}


\begin{description}
  \item[na mesma linha:] \$ E=mc\^{}2 \$
  \item[como parágrafo:] \textbackslash[ E=mc\^{}2 \textbackslash]
\end{description}


\vspace{\baselineskip}


\$ \textbackslash{}mathbb\{R\} \$\enspace $\rightarrow \mathbb{R}$

\$ \textbackslash{}text\{texto normal\} \$\enspace $\rightarrow \text{texto normal}$

\$ \textbackslash{}mathit\{nomes\_longos\} \$\enspace $\rightarrow \mathit{nomes\_longos}$


\section*{Floats}

\textbackslash{}begin\{figure\}

\quad\textbackslash{}centering

\quad\textbackslash{}includegraphics[width=0.8\textbackslash{}textwidth]\{arquivo\}

\quad\textbackslash{}caption\{Legenda\textbackslash{}label\{nomesimpatico\}\}

\textbackslash{}end\{figure\}


\vspace{\baselineskip}

\textbackslash{}begin\{table\}\quad \mbox{(veja \url{tablesgenerator.com}
                                     e \textsf{texdoc booktabs})}

\quad\textbackslash{}centering

\quad\textbackslash{}begin\{tabular\}

\quad\quad\dots

\quad\textbackslash{}end\{tabular\}

\quad\textbackslash{}caption\{Legenda\textbackslash{}label\{nomesimpatico\}\}

\textbackslash{}end\{table\}

\columnbreak


\section*{Estrutura}

\textbackslash{}begin\{itemize\}\quad (\,ou \textbackslash{}begin\{enumerate\}\,)

\quad\textbackslash{}item Algum texto\dots

\textbackslash{}end\{itemize\}\quad (\,ou \textbackslash{}end\{enumerate\}\,)


\vspace{\baselineskip}


\textbackslash{}begin\{description\}

\quad\textbackslash{}item[termo] descrição ou discussão

\textbackslash{}end\{description\}


\vspace{\baselineskip}


\textbackslash{}begin\{verse\}\quad (veja também a \textit{package} \textsf{verse})

\quad verso\dots\ \textbackslash\textbackslash

\quad verso\dots\ \textbackslash\textbackslash

\textbackslash{}end\{verse\}


\vspace{\baselineskip}


\textbackslash{}begin\{quotation\}

\quad citação\dots

\textbackslash{}end\{quotation\}


\vspace{\baselineskip}


\begin{description}
    \item[\textsc{url}s:] \textbackslash{}url\{endereço\}
\end{description}

\section*{Refs cruzadas/citações}

\textbackslash{}label\{nomesimpatico\}

\textbackslash{}ref\{nomesimpatico\},
\textbackslash{}pageref\{nomesimpatico\}


\vspace{\baselineskip}


\begin{description}
\item[citação simples:]~\vspace{2pt}\newline
    \null\quad\textbackslash{}cite[p.~25]\{fulano\} $\Rightarrow$ Fulano de Tal, 1987, p.~25\vspace{6pt}

  \item[citação no texto:]~\vspace{2pt}\newline
    \null\quad\textbackslash{}cite\textbf{t}[p.~25]\{fulano\} $\Rightarrow$ Fulano de Tal (1987, p.~25)\vspace{6pt}

  \item[citação entre parênteses:]~\vspace{2pt}\newline
    \null\quad\textbackslash{}cite\textbf{p}[p.~25]\{fulano\} $\Rightarrow$ (Fulano de Tal, 1987, p.~25)
\end{description}

\section*{Colunas}

\textbackslash{}begin\{multicols\}\{2\}\quad (\textit{package} \textsf{multicols})

\quad texto em colunas\dots

\textbackslash{}end\{multicols\}

\columnbreak

\section*{Línguas}

\textbackslash{}br, \textbackslash{}en

\textbackslash{}textbr\{texto\}, \textbackslash{}texten\{texto\}

(define línguas em \textsf{\textbackslash{}documentclass} e \textsf{\textbackslash{}babeltags})

\section*{Caracteres especiais}

\begin{description}
  \item[elipse:] \textbackslash{}dots\{\}
  \item[espaço não-separável:] \textasciitilde{} (não quebra a linha)
  \item[espaço aumentado:] \textbackslash{}enspace\{\} ou \textbackslash{}quad\{\}
  % \strut -> https://github.com/schlcht/microtype/issues/10
  \item[aspas tipográficas:] \strut\`\space\,\`\space\;\!texto\;\!\textquotesingle\:\textquotesingle,
                \`\space\;\!texto\;\!\textquotesingle
  \item[hífen, traço, travessão:] -\quad -\hspace{.7pt}-\quad -\hspace{.7pt}-\hspace{.7pt}-
  \item[feminino, grau, ordinal:]
                Prof.\textsuperscript{a} (\textbackslash textsuperscript\{a\}),\\
                1\textordfeminine\ (\textbackslash textordfeminine),
                90\textdegree\ (\textbackslash textdegree),
                2\textordmasculine\ (\textbackslash textordmasculine)
\end{description}

\vspace{\baselineskip}

\begin{description}
  \item[não podem ser digitados diretamente:]~\vspace{5pt}\newline
    \textbackslash\#
    \quad\textbackslash\$
    \quad\textbackslash\%
    \quad\textbackslash\&
    \quad\textbackslash\_
    \quad\textbackslash{}textbackslash\{\}:\enspace\textbackslash
    \vspace{3pt}\newline
    \textbackslash\{
    \quad\textbackslash\}
    \quad\textbackslash{}textquotesingle\{\}:\enspace\textquotesingle
    \quad\textbackslash{}textquotedbl\{\}:\enspace\textquotedbl
    \vspace{3pt}\newline
    \textbackslash{}textasciitilde\{\}:\enspace\textasciitilde
    \quad\textbackslash{}textasciicircum\{\}:\enspace\textasciicircum
    \quad\raisebox{-.1\baselineskip}{\`\space}%
        \hspace{.12em}\textbackslash{}space\{\}:
        \enspace\raisebox{-.1\baselineskip}{\`\space}
\end{description}


\vspace{\baselineskip}


\begin{description}
  \item[lista de símbolos:] \textsf{texdoc symbols-a4}
  \item[busca símbolos:] \url{detexify.kirelabs.org/classify.html}
\end{description}

\section*{Formatação manual}

\textbackslash{}emph\{\emph{enfatizado}\} (em geral, \textit{itálico})

\textbackslash{}textit\{\textit{itálico}\}

\textbackslash{}textbf\{\textbf{negrito}\}

\textbackslash{}textsc\{\textsc{Versalete}\}

\textbackslash{}textsf\{\textsf{Fonte sem serifa}\}

\textbackslash{}texttt\{\texttt{Fonte terminal}\}

\textbackslash{}textsuperscript\{\textsuperscript{sobrescrito}\}

\textbackslash{}textsubscript\{\textsubscript{subscrito}\}

\textbackslash{}- (sugere hifenização)

\textbackslash{}linebreak (sugere divisão)

\textbackslash{}pagebreak (sugere divisão)

\textbackslash{}newline ou \textbackslash\textbackslash{} (força quebra)

\textbackslash{}newpage (força quebra)

\textbackslash{}noindent (inicia parágrafo sem indentação na
                         1\textordfeminine\ linha)

\textbackslash{}begin\{center|flushleft|flushright\}\quad \mbox{(veja também
                                                    \textsf{texdoc ragged2e})}

\quad texto centralizado | alinhado à esquerda | à direita

\textbackslash{}end\{center|flushleft|flushright\}

\end{multicols}

\printbibliography[
  title=\refname\label{bibliografia}, % "Referências", recomendado pela ABNT
  %title=\bibname\label{bibliografia}, % "Bibliografia"
]

\end{document}
